\documentclass[xcolor=pdftex,dvipsnames,table,mathserif,aspectratio=169]{beamer}
\usetheme{metropolis}
%\usetheme{Darmstadt}
%\usepackage{times}
%\usefonttheme{structurebold}

\usepackage[english]{babel}
%\usepackage[table]{xcolor}
\usepackage{pgf,pgfarrows,pgfnodes,pgfautomata,pgfheaps}
\usepackage{amsmath,amssymb,setspace}
\usepackage[latin1]{inputenc}
\usepackage[T1]{fontenc}
\usepackage{relsize}
\usepackage[absolute,overlay]{textpos} 
\newenvironment{reference}[2]{% 
  \begin{textblock*}{\textwidth}(#1,#2) 
      \footnotesize\it\bgroup\color{red!50!black}}{\egroup\end{textblock*}} 

\DeclareMathSizes{10}{10}{6}{6} 


\title{Differentiated Bertrand}
\author{C.Conlon}
\institute{Grad IO }
\date{Fall 2020}
\setbeamerfont{equation}{size=\tiny}
\begin{document}

\frame{\titlepage}

\begin{frame}
\frametitle{Differentiated Products Bertrand}
\small
Consider the multi-product Bertrand problem where firms solve: $\arg \max_{p \in \mathcal{J}_f} \pi_f (\mathbf{p}) = \sum_{j \in \mathcal{J}_f} (p_j - c_j) \cdot q_j(\mathbf{p})$:
\begin{align*}
 0&= q_j(\mathbf{p}) + \sum_{k \in \mathcal{J}_f} (p_k - c_k) \frac{\partial q_{k}}{\partial p_j}(\mathbf{p}) \\
\rightarrow p_j &=q_{j}(\mathbf{p}) \left[-\frac{\partial q_{j}}{\partial p_{j}}(\mathbf{p})\right]^{-1} + c_{j} + \sum_{k \in \mathcal{J}_{f} \setminus j} \left(p_{k}-c_{k}\right) \underbrace{\frac{\partial q_{k}}{\partial p_{j}}(\mathbf{p})\left[-\frac{\partial q_{j}}{\partial p_{j}}(\mathbf{p})\right]^{-1}}_{D_{jk}(\mathbf{p})}\\
p_j(p_{-j}) &= \underbrace{\frac{1}{1+1/\epsilon_{jj}(\mathbf{p})}}_{\text{Markup}} \left[ c_j + \sum_{k \in \mathcal{J}_{f} \setminus j}  (p_k-c_k) \cdot  D_{jk} (\mathbf{p}) \right].
\end{align*}
We call $D_{jk}(\mathbf{p}) = \frac{\frac{\partial q_{k}}{\partial p_j}(\mathbf{p})}{\left| \frac{\partial q_{j}}{\partial p_j}(\mathbf{p}) \right|}$ the \alert{diversion ratio}.
\end{frame}

\begin{frame}
\frametitle{Differentiated Products Bertrand}
We can also re-write the best-response in the Lerner Index form:
\begin{align*}
p_j(p_{-j}) &= \underbrace{\frac{1}{1+1/\epsilon_{jj}(\mathbf{p})}}_{\text{Markup}} \left[c_j + \underbrace{\sum_{k \in \mathcal{J}_{f} \setminus j}  (p_k-c_k) \cdot  D_{jk} (\mathbf{p}) }_{\text{opportunity cost}}\right]
\end{align*}

\end{frame}


\begin{frame}
\frametitle{Differentiated Products Bertrand}
It is helpful to define the matrix $\Delta$ with entries:
\begin{eqnarray*}
\Delta_{(j,k)}(\mathbf{p}) = \left\{\begin{array}{lr}
         - \frac{\partial q_{j}}{\partial p_k}(\mathbf{p}) & \text{for }  (j,k) \in \mathcal{J}_f\\
       	  \quad 0 & \text{for } (j,k) \notin \mathcal{J}_f
        \end{array} \right\}
\end{eqnarray*}
We can re-write the FOC in matrix form:
\begin{eqnarray*}
q(\mathbf{p}) = \Delta(\mathbf{p})\cdot(\mathbf{p}-\mathbf{mc})
\end{eqnarray*}
\end{frame}


\begin{frame}
\frametitle{What do we want to learn from demand systems?}
We can recover markups and marginal costs:
\begin{eqnarray*}
(\mathbf{p}-\mathbf{mc}) =  \Delta(\mathbf{p})^{-1} q(\mathbf{p})  \Rightarrow \mathbf{mc}= \mathbf{p} - \Delta(\mathbf{p})^{-1} q(\mathbf{p})
\end{eqnarray*}
\end{frame}

\begin{frame}
\frametitle{What do we want to learn from demand systems?}
\begin{itemize}
\item We need two objects from the demand system:
\begin{itemize}
\item The (vector of) \alert{predicted sales} at the (vector of) prices $\mathbf{q(p)}$.
\item The \alert{derivatives} (or elasticities) of the demand function  $\frac{\partial q_k}{\partial p_j}(\mathbf{p})$ 
\item Or the \alert{diversion ratios}: $D_{jk}(\mathbf{p})$.
\item Welfare is often related to substitution to no purchase or ``outside goods''.
\end{itemize}
\item Ideally, these objects depend on the full vector of prices $\mathbf{p}$.
\begin{itemize}
\item We can impose restrictions on the demand curve, such as constant slope or constant elasticity.
\end{itemize}
\item With these in hand, we can:
\begin{itemize}
\item Recover estimates of marginal cost (remember these are proprietary firm information and accounting estimates are often unreliable).
\end{itemize}
\end{itemize}
\end{frame}


\begin{frame}
\frametitle{Overview}
Multiproduct Demand system estimation is probably the most important contribution from the New Empirical IO literature
\begin{itemize}
\item Demand is an important primitive (think about Econ 101)
\item Welfare Analysis, Pass through of taxation, Value of Advertising, Price Effects of Mergers all rely on demand estimates.
\item Last 5-10 years has seen successful export to other fields: Trade, Healthcare, Education, Urban Economics, Marketing, Operations Research.
\item Anywhere consumers face a number of options and ``prices'': doctors, hospitals, schools, mutual funds, potential dates, etc.
\item There are many many more applications.
\end{itemize}
\end{frame}

\begin{frame}
\frametitle{Two Major Issues}
\begin{itemize}
\item Endogeneity of Prices
\begin{itemize}
\item Prices are not randomly determined, but set strategically by firms who observe the demand curve
\item The \alert{simultaneity} of supply and demand creates a problem. We see the market clearing $(P^*,Q^*)$ over several periods, but in general we do not know which curve shifted.
\end{itemize}
\item Multiple Products/Flexibility
\begin{itemize}
\item We want to allow for flexible (data driven) substitution across products but if we have $J$ products, then $\partial Q_j / \partial P_k$ might have $J^2$ elements.
\item We may also think that $\partial Q_j / \partial P_k$ varies with $P$ and with other covariates $x$.
\item We may also care about $\partial^2 Q_j / \partial P_k^2$, $\partial^3 Q_j / \partial P_k^3$ and so on.
\end{itemize}
\item We will address the issues separately and then see how to put them back together.
\end{itemize}
\end{frame}

\begin{frame}
\frametitle{Taxonomy of Demand Systems}
\begin{itemize}
\item Representative Consumer vs. Heterogeneous Agents?
\item Discrete Choices vs. Continuous Choices?
\item Single Product vs. Many Products?
\item Product Space vs. Characteristic Space?
\begin{itemize}
\item Do consumers choose products in product space or in characteristic space?
\end{itemize}
\end{itemize}
\end{frame}


\begin{frame}
\frametitle{Data Sources}
\begin{itemize}
\item We can either have \alert{aggregate data} (market level data on $(P_j,Q_j)$)
\begin{itemize}
\item Many supermarket ``scanner' datasets: Nielsen, IRI, Dominick's
\item NPD: video games/computers/consumer electronics.
\item Many proprietary single firm sources
\end{itemize}
\item or \alert{micro data} panel of data with same individuals over time.
\begin{itemize}
\item Best example is Nielsen Homescan Consumer Panel data.
\item Visa/Mastercard datasets
\item Medicare
\end{itemize}
\item Sometimes we have a combination of both
\begin{itemize}
\item Often we have aggregate purchase data plus some \alert{micro data} from a survey on a subpopulation.
\item ie: asking people who purchased GM cars which other cars they were considering.
\item Using scanner data for alcohol purchases and comparing to consumption surveys by income and education.
\end{itemize}
\end{itemize}
\end{frame}


\end{document}
