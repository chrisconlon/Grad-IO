\def\beamerclassoptions{[xcolor=pdftex,dvipsnames,table,mathserif]}
% -----------------------------------------------------------------------------
% Thin wrapper preamble for Grad-IO slides
% Usage:
%  - Optionally set \def\beamerclassoptions{[<opts>]} before \input-ing this file
%    so a slide can control per-file beamer options (e.g. [handout,aspectratio=169]).
%  - This file ONLY issues the \documentclass once (honoring \beamerclassoptions)
%    and then loads the shared package `gradio-preamble.sty` which contains the
%    guarded package loads and macro definitions.
% -----------------------------------------------------------------------------
\ifdefined\beamerclassoptions
        % Expand \beamerclassoptions (which should be like "[notes=show]") safely
        \begingroup\edef\x{\endgroup\noexpand\documentclass\beamerclassoptions{beamer}}\x
\else
        \documentclass[handout,10pt,aspectratio=169]{beamer}
\fi

% Load the canonical shared preamble package. Try the local resources folder first
% so slide-level inputs continue to work in-place; if installed system-wide then
% \RequirePackage will also find it by name.
\IfFileExists{./gradio-preamble.sty}{\RequirePackage{./gradio-preamble}}{%
    \IfFileExists{gradio-preamble.sty}{\RequirePackage{gradio-preamble}}{%
        \RequirePackage{gradio-preamble}% final fallback to system-installed package
    }
}

% Any slide-specific packages or macros should be declared in the slide file
% after `% -----------------------------------------------------------------------------
% Thin wrapper preamble for Grad-IO slides
% Usage:
%  - Optionally set \def\beamerclassoptions{[<opts>]} before \input-ing this file
%    so a slide can control per-file beamer options (e.g. [handout,aspectratio=169]).
%  - This file ONLY issues the \documentclass once (honoring \beamerclassoptions)
%    and then loads the shared package `gradio-preamble.sty` which contains the
%    guarded package loads and macro definitions.
% -----------------------------------------------------------------------------
\ifdefined\beamerclassoptions
        % Expand \beamerclassoptions (which should be like "[notes=show]") safely
        \begingroup\edef\x{\endgroup\noexpand\documentclass\beamerclassoptions{beamer}}\x
\else
        \documentclass[handout,10pt,aspectratio=169]{beamer}
\fi

% Load the canonical shared preamble package. Try the local resources folder first
% so slide-level inputs continue to work in-place; if installed system-wide then
% \RequirePackage will also find it by name.
\IfFileExists{./gradio-preamble.sty}{\RequirePackage{./gradio-preamble}}{%
    \IfFileExists{gradio-preamble.sty}{\RequirePackage{gradio-preamble}}{%
        \RequirePackage{gradio-preamble}% final fallback to system-installed package
    }
}

% Any slide-specific packages or macros should be declared in the slide file
% after `% -----------------------------------------------------------------------------
% Thin wrapper preamble for Grad-IO slides
% Usage:
%  - Optionally set \def\beamerclassoptions{[<opts>]} before \input-ing this file
%    so a slide can control per-file beamer options (e.g. [handout,aspectratio=169]).
%  - This file ONLY issues the \documentclass once (honoring \beamerclassoptions)
%    and then loads the shared package `gradio-preamble.sty` which contains the
%    guarded package loads and macro definitions.
% -----------------------------------------------------------------------------
\ifdefined\beamerclassoptions
        % Expand \beamerclassoptions (which should be like "[notes=show]") safely
        \begingroup\edef\x{\endgroup\noexpand\documentclass\beamerclassoptions{beamer}}\x
\else
        \documentclass[handout,10pt,aspectratio=169]{beamer}
\fi

% Load the canonical shared preamble package. Try the local resources folder first
% so slide-level inputs continue to work in-place; if installed system-wide then
% \RequirePackage will also find it by name.
\IfFileExists{./gradio-preamble.sty}{\RequirePackage{./gradio-preamble}}{%
    \IfFileExists{gradio-preamble.sty}{\RequirePackage{gradio-preamble}}{%
        \RequirePackage{gradio-preamble}% final fallback to system-installed package
    }
}

% Any slide-specific packages or macros should be declared in the slide file
% after `\input{.../resources/preamble.tex}`.

`.

`.


\usetheme{default}
%\usetheme{Darmstadt}
%\usepackage{times}
%\usefonttheme{structurebold}

\usepackage[english]{babel}
%\usepackage[table]{xcolor}
\usepackage{pgf,pgfarrows,pgfnodes,pgfautomata,pgfheaps}
\usepackage{amsmath,amssymb,setspace,centernot}
\usepackage[latin1]{inputenc}
\usepackage[T1]{fontenc}
\usepackage{relsize}
\usepackage{pdfpages}
\usepackage[absolute,overlay]{textpos} 


\newenvironment{reference}[2]{% 
  \begin{textblock*}{\textwidth}(#1,#2) 
      \footnotesize\it\bgroup\color{red!50!black}}{\egroup\end{textblock*}} 

\DeclareMathSizes{10}{10}{6}{6} 

\begin{document}
\title{Part 7: Multinomial Choice}
\author{Chris Conlon}
\institute{Microeconometrics}
\date{\today}

\frame{\titlepage}

\section{Intro}

\begin{frame}
\frametitle{Motivation}
Most decisions agents make are not necessarily binary:
\begin{itemize}
\item Choosing a level of schooling (or a major).
\item Choosing an occupation.
\item Choosing a partner.
\item Choosing where to live.
\item Choosing a brand of (yogurt, laundry detergent, orange juice, cars, etc.).
 \end{itemize}
\end{frame}

\begin{frame}
\frametitle{Setup}
We consider a \alert{multinomial discrete choice}:
\begin{itemize}
\item in period $t$
\item with $J_t$ alternatives.
\item subscript individual agents by $i$.
\item agents choose $j \in J_t$ with probability $P_{ijt}$.
\item Agent $i$ receives utility $U_{ij}$ for choosing $j$.
\item Choice is exhaustive and mutually exclusive.
 \end{itemize}\pause
Consider the simple example $(t=1)$:
\begin{eqnarray*}
P_{ij} = Prob( U_{ij} > U_{ik} \quad \forall j \neq k)
\end{eqnarray*}
\end{frame}

\begin{frame}
\frametitle{Setup}
Now consider separating the utility into the observed $V_{ij}$ and unobserved components $\varepsilon_{ij}$.
\begin{eqnarray*}
P_{ij} &=& Prob( U_{ij} > U_{ik} \quad \forall j \neq k)\\
 &=& Prob( V_{ij} + \varepsilon_{ij} > V_{ik} + \varepsilon_{ij} \quad \forall j \neq k)\\
 &=& Prob( \varepsilon_{ij}-\varepsilon_{ik} > V_{ik} - V_{ij} \quad \forall j \neq k)
\end{eqnarray*}
\pause
It is helpful to define $f(\varepsilon_{i})$ as the $J$ vector of individual $i$'s unobserved utility.
\begin{eqnarray*}
P_{ij} &=& Prob( \varepsilon_{ij}-\varepsilon_{ik} > V_{ik} - V_{ij} \quad \forall j \neq k)\\
&=& \int I( \varepsilon_{ij}-\varepsilon_{ik} > V_{ik} - V_{ij} ) f( \varepsilon_i) \partial \varepsilon_i \\
\end{eqnarray*}
\end{frame}

\begin{frame}
\frametitle{Setup}
In order to compute the choice probabilities, we must perform a $J$ dimensional integral over $f(\varepsilon_i)$.
\begin{eqnarray*}
P_{ij} &=&  \int I( \varepsilon_{ij}-\varepsilon_{ik} > V_{ik} - V_{ij} ) f( \varepsilon_i) \partial \varepsilon_i 
\end{eqnarray*}
There are some choices that make our life easier
\begin{itemize}
\item Multivariate normal: $\varepsilon_i  \sim N(0,\Omega)$. $\longrightarrow$ \alert{ multinomial probit}.
\item Gumbel/Type 1 EV: $f(\varepsilon_i) = e^{-\varepsilon_{ij}}  e^{-e^{-\varepsilon_{ij}}}  $ and $F(\varepsilon_i) = 1- e^{-e^{-\varepsilon_{ij}}}$ $\longrightarrow$ \alert{multinomial logit}
\item There are also heteroskedastic variants of the Type I EV/ Logit framework.
\end{itemize}
\end{frame}

\begin{frame}
\frametitle{Errors}
Allowing for full support $[-\infty, \infty]$ errors provide two key features:
\begin{itemize}
\item Smoothness: $P_{ij}$ is everywhere continuously differentiable in $V_{ij}$.
\item Bound $P_{ij} \in (0,1)$ so that we can rationalize any observed pattern in the data.
\item What does $\varepsilon_{ij}$ really mean? (unobserved utility, idiosyncratic tastes, etc.)
\end{itemize}
\end{frame}

\begin{frame}
\frametitle{Basic Identification}
\small
\begin{itemize}
\item Only differences in utility matter: $Prob( \varepsilon_{ij}-\varepsilon_{ik} > V_{ik} - V_{ij} \quad \forall j \neq k)$
\item Adding constants is irrelevant: if $U_{ij} > U_{ik}$ then $U_{ij} + a > U_{ik} + a$.
\item Only differences in alternative specific constants can be identified
\begin{eqnarray*}
U_b &=& X_b \beta + k_b  + \varepsilon_b\\
U_c &=& X_c \beta + k_c  + \varepsilon_c
\end{eqnarray*}
only $d = k_b - k_c$ is identified.
\item This means that we can only include $J-1$ such $k$'s and need to normalize one to zero. (Much like fixed effects).
\item We cannot have individual specific factors that enter the utility of all options such as income $\theta Y_i$. We can allow for interactions between individual and choice characterstics $\theta p_{j}/ Y_i$.
\end{itemize}
\end{frame}

\begin{frame}
\frametitle{Basic Identification}
Location
\begin{itemize}
\item Technically we can't really fully specify $f(\varepsilon_i)$ since we can always re-normalize: $\widetilde{\varepsilon_{ijk}} = \varepsilon_{ij} - \varepsilon_{ik}$ and write $g(\widetilde{\varepsilon_{ik}})$. Thus any $g(\widetilde{\varepsilon_{ik}})$ is consistent with infinitely many $f(\varepsilon_i)$.
\item Logit pins down $f(\varepsilon_i)$ sufficiently with parametric restrictions.
\item Probit does not. We must generally normalize one dimension of $f(\varepsilon_i)$ in the probit model. Usually a diagonal term of $\Omega$ so that $\omega_{11} =1$ for example. (Actually we need to do more!).
\end{itemize}
Scale
\begin{itemize}
\item Consider: $U_{ij}^0 = V_{ij} + \varepsilon_{ij}$ and  $U_{ij}^1 = \lambda V_{ij} + \lambda \varepsilon_{ij}$ with $\lambda > 0$. Multiplying by constant $\lambda$ factor doesn't change any statements about $U_{ij} > U_{ik}$.
\item We normalize this by fixing the variance of $\varepsilon_{ij}$ since $Var(\lambda \varepsilon_{ij} ) = \sigma_e^2 \lambda^2$.
\item Normalizing this variance normalizes the scale of utility.
\item For the logit case the variance is normalized to $\pi^2/6$. (this emerges as a constant of integration to guarantee a proper density).
\end{itemize}
\end{frame}

\begin{frame}
\frametitle{Observed Heteroskedasticity}
Consider the case where $Var(\varepsilon_{ij}^B) = \sigma^2$ and   $Var(\varepsilon_{ij}^C) =  k^2 \sigma^2$ :
\begin{itemize}
\item We can estimate
\begin{eqnarray*}
U_{ij} &=& x_j \beta + \varepsilon_{ij}^B\\
U_{ij} &=& x_j \beta + \varepsilon_{ij}^C
\end{eqnarray*}
becomes:
\begin{eqnarray*}
U_{ij} &=& x_j \beta + \varepsilon_{ij}\\
U_{ij} &=& x_j \beta/k+ \varepsilon_{ij}
\end{eqnarray*}
\item Some interpret this as saying that in segment $C$ the unobserved factors are $\hat{k}$ times larger.
\end{itemize}
\end{frame}

\begin{frame}
\frametitle{Deeper Identification Results}
Different ways to look at identification
\begin{itemize}
\item Are we interested in non-parametric identification of $V_{ij}$, specifying $f(\varepsilon_i)$?
\item Or are we interested in non-parametric identification of $U_{ij}$. (Generally hard).
\begin{itemize}
\item Generally we require a large support (special-regressor) or ``completeness'' condition.
\item Lewbel (2000) does random utility with additively separable but nonparametric error.\item Berry and Haile (2015) with non-separable error (and endogeneity).
\end{itemize}
\end{itemize}
\end{frame}


\begin{frame}
\frametitle{Logit}
\begin{itemize}
\item Logit has closed form choice probabilities
\begin{eqnarray*}
P_{ij} = \frac{e^{V_{ij}}}{\sum_k e^{V_{ik}}} \approx \frac{e^{\beta' x_{ij}}}{\sum_k e^{\beta' x_{ik}}}
\end{eqnarray*}
\item Approximation arises from the hope that we can approximate $V_{ij} \approx  X_{ik} \beta$ with something linear in parameters.
\item Expected maximum also has closed form:
\begin{eqnarray*}
E[\max_j U_{ij}] = \log \left(\sum_j \exp[V_{ij}] \right) + C
\end{eqnarray*}
\end{itemize}
\end{frame}



\begin{frame}
\frametitle{Logit Inclusive Value}
\begin{itemize}
\item Logit Inclusive Value is helpful for several reasons
\begin{eqnarray*}
E[\max_j U_{ij}] = \log \left(\sum_j \exp[V_{ij}] \right) + C
\end{eqnarray*}
\item Expected utility of best option (without knowledge of realized $\varepsilon_i$) does not depend on $\epsilon_{ij}$.
\item This is a globally concave function in $V_{ij}$ (more on that later).
\item Allows simple computation of $\Delta CS$ for consumer welfare.
\end{itemize}
\end{frame}

\begin{frame}
\frametitle{Alternative Interpretation}
Statistics/Computer Science offer an alternative interpretation
\begin{itemize}
\item Sometimes this is called \alert{softmax} regression.
\item Think of this as a continuous/concave approximation to the maximum.
\item Consider $\max\{x,y\}$ vs $\log(\exp(x) + \exp(y))$. The $\exp$ exaggerates the differences between $x$ and $y$ so that the larger term dominates.
\item We can accomplish this by rescaling $k$:  $\log(\exp(kx) + \exp(ky))/k$ as $k$ becomes large the derivatives become infinite and this approximates the ``hard'' maximum.
\item $g(1, 2) = 2.31$, but $g(10, 20) = 20.00004$.
\end{itemize}
\end{frame}

\begin{frame}{Alternative Interpretation}
\begin{figure}[htbp]
\begin{center}
\includegraphics[width=2in]{hardmax.png}
\includegraphics[width=2in]{softmax.png}
\end{center}
\end{figure}
\end{frame}

\begin{frame}{Back to Scale of Utility}
\begin{itemize}
\item Consider $U_{ij}^{*} = V_{ij} + \varepsilon_{ij}^{*}$ with $Var(\varepsilon^{*}) = \sigma^2 \pi^2/6$.
\item Without changing behavior we can divide by $\sigma$ so that $U_{ij} = V_{ij}/\sigma + \varepsilon_{ij}$ and $Var(\varepsilon^{*}/\sigma)=Var(\varepsilon) = \pi^2/6$
\begin{eqnarray*}
P_{ij} = \frac{e^{V_{ij}/\sigma}}{\sum_k e^{V_{ik}/\sigma}} \approx \frac{e^{\beta^{*}/\sigma \cdot x_{ij}}}{\sum_k e^{\beta^{*}/\sigma \cdot x_{ik}}}
\end{eqnarray*}
\item Every coefficient $\beta$ is rescaled by $\sigma$. This implies that only the ratio $\beta^{*}/\sigma$ is identified. 
\item Coefficients are relative to variance of unobserved factors. More unobserved variance $\longrightarrow$ smaller $\beta$.
\item Ratio $\beta_1/\beta_2$ is invariant to the scale parameter $\sigma$.
\end{itemize}
\end{frame}

\begin{frame}{Taste Variation}
\begin{itemize}
\item Logit allows for taste variation across individuals if two conditions are met: \alert{individual level data} and \alert{interact observed characterstics} only.
\item We often want to allow for something like $U_{ij} = x_{j} \beta_i - \alpha_i p_j + \varepsilon_{ij}$. 
\item We might want $\beta_i = \theta / y_i$ where $y_i$ is the income for individual $i$ or $\beta_i = \theta y_i$, etc.
\item Can also have $z_{ij}$ such as the distance between $i$ and hospital $j$.
\item Cannot have unobserved heterogeneity or heteroskedasticity in $\varepsilon_{ij}$.
\end{itemize}
\end{frame}

\begin{frame}{Taste Variation}
\begin{eqnarray*}
\frac{P_{ij}}{P_{ik}} = \frac{e^{V_{ij}}}{\sum_{k'} e^{V_{ik'}}} / \frac{e^{V_{ik}}}{\sum_{k'} e^{V_{ik'}}} = \frac{e^{V_{ij}}}{e^{V_{ik}}} = \exp[V_{ij} - V_{ik}].
\end{eqnarray*}
\begin{itemize}
\item The ratio of choice probabilities for $j$ and $k$ depends only on $j$ and $k$ and not on any alternative $l$, this is known as \alert{independence of irrelevant alternatives}.
\item For some (Luce (1959)) IIA was an attractive property for axiomatizing choice.
\item In fact the logit was derived in the search for a statistical model that satsified various axioms.
\end{itemize}
\end{frame}

\begin{frame}{IIA Property}
\begin{itemize}
\item The well known counterexample: You can choose to go to work on a car $c$ or blue bus $bb$. $P_{c} = P_{bb} = \frac{1}{2}$ so that $\frac{P_c}{P_{bb}} = 1$.
\item Now we introduce a red bus $rb$ that is identical to $bb$. Then $\frac{P_{rb}}{P_{bb}} = 1$ and $P_{c} = P_{bb}= P_{rb} = \frac{1}{3}$ as the logit model predicts.
\item In reality we don't expect painting a bus red would change the number of individuals who drive a car so we would anticipate $P_{c} = \frac{1}{2}$ and $P_{bb} = P_{rb} = \frac{1}{4}$.
\item We may not encounter too many cases where $\rho_{\varepsilon_{ik},\varepsilon_{ij}} \approx 1$, but we have many cases where this $\rho_{\varepsilon_{ik},\varepsilon_{ij}} \neq 0$
\item What we need is the ratio of probabilities to change when we introduce a third option!
\end{itemize}
\end{frame}

\begin{frame}{IIA Property}
\begin{itemize}
\item IIA implies that we can obtain consistent estimates for $\beta$ on any subset of alternatives.
\item This means instead of using all $J$ alternatives in the choice set, we could estimate on some subset $S \subset J$.
\item This used to be a way to reduce the computational burden of estimation (not clear this is an issue in 2016).
\item Sometimes we have \alert{choice based samples} where we oversample people who choose a particular alternative. Manski and Lerman (1977) show we can get consistent estimates for all but the ASC. This requires knowledge of the difference between the true rate $A_j$ and the choice-based sample rate $S_j$.
\item Hausman proposes a specification test of the logit model: estimate on the full dataset to get $\hat{\beta}$, construct a smaller subsample $S^k \subset J$ and $\hat{\beta^k}$ for one or more subsets $k$. If $|\hat{\beta}^k - \hat{\beta}|$ is small enough.
\end{itemize}
\end{frame}

\begin{frame}{IIA Property}
\begin{eqnarray*}
\frac{\partial P_{ij}}{\partial z_{ij}} = P_{ij}(1- P_{ij}) \frac{\partial V_{ij}}{\partial z_{ij}}
\end{eqnarray*}
And Elasticity:
\begin{eqnarray*}
\frac{ \partial \log P_{ij}}{ \partial \log z_{ij}} = P_{ij}(1- P_{ij}) \frac{\partial V_{ij}}{\partial z_{ij}} \frac{z_{ij}}{P_{ij}} = (1- P_{ij}) z_{ij} \frac{\partial V_{ij}}{\partial z_{ij}}
\end{eqnarray*}
With cross effects:
\begin{eqnarray*}
\frac{\partial P_{ij}}{\partial z_{ik}} = -P_{ij} P_{ik} \frac{\partial V_{ik}}{\partial z_{ik}}
\end{eqnarray*}
And Elasticity:
\begin{eqnarray*}
\frac{ \partial \log P_{ij}}{ \partial \log z_{ik}} = -P_{ik} z_{ik} \frac{\partial V_{ik}}{\partial z_{ik}}
\end{eqnarray*}
For the linear $V_{ij}$ case we have that $\frac{\partial V_{ij}}{\partial z_{ij}}=  \beta_z$.
\end{frame}

\begin{frame}{Proportional Substitution}
Cross elasticity doesn't really depend on $j$.
\begin{eqnarray*}
\frac{ \partial \log P_{ij}}{ \partial \log z_{ik}} = -P_{ik} z_{ik} \underbrace{\frac{\partial V_{ik}}{\partial z_{ik}}}_{\beta_z}.
\end{eqnarray*}
\begin{itemize}
\item This leads to the idea of proportional substitution. As option $k$ gets better it proportionally reduces the shares of the all other choices.
\item Likewise removing an option $k$ means that $\tilde{P}_{ij} = \frac{P_{ij}}{1-P_{ik}}$ for all other $j$.
\item This might be a desirable property but probably not.
\end{itemize}
\end{frame}



\begin{frame}{Alternatives to IIA}
IIA doesn't seem like a particularly desireable property. How can we relax it?
\begin{itemize}
\item The problem arises because we required that $\varepsilon_{ij}$ were IID.
\item We would like to allow for a more general heteroskedastic structure on $\varepsilon_{ij}$
\item Options
\begin{itemize}
\item Lots of observable individual characteristics and interact them with the $x_j$'s. Then maybe IIA is mostly about smoothness not about undesireable properties.
\item Multivariate Probit allows for $\varepsilon_i \sim N(0,\Omega)$.
\item Put some more structure on the problem: Assume that $\varepsilon_i$ has a block structure. Assign choices to categories and allow for more correlation between choices in prespecified categories.
\item Mixed logit allows for unobserved heterogeneity $\nu_{i}$ that we can interact with $x_j$ from some arbitrary distribution $f(\nu_i | \theta)$. Conditional on a $\nu_i$ individual behavior is still logit. This is actually a basis structure on $\varepsilon_i$.
\end{itemize}
\end{itemize}
\end{frame}

\begin{frame}{Nested Logit}
A traditional (and simple) relaxation of the IIA property is the Nested Logit. This model is often presented as two sequential decisions.
\begin{itemize}
\item First consumers choose a category (following an IIA logit).
\item Within a category consumers make a second decision (following the IIA logit).
\item This leads to a situation where while choices within the same nest follow the IIA property (do not depend on attributes of other alternatives) choices among different nests do not!
\end{itemize}
\end{frame}

\begin{frame}{Alternative Interpretation}
\begin{figure}[htbp]
\begin{center}
\includegraphics[width=4in]{nesting.png}
\end{center}
\end{figure}
\end{frame}

\begin{frame}{Nested Logit}
Utility looks basically the same as before:
\begin{eqnarray*}
U_{ij} = V_{ij} + \underbrace{\eta_{ig} + \widetilde{\varepsilon_{ij}}}_{\varepsilon_{ij}(\lambda_g)}
\end{eqnarray*}
\begin{itemize}
\item We add a new term that depends on the group $g$ but not the product $j$ and think about it as varying unobservably over individuals $i$ just like $\varepsilon_{ij}$.
\item Now $\varepsilon_i \sim F(\varepsilon)$ where $F(\varepsilon) = \exp[-\sum_{g=G}^G \left(\sum_{j \in J_g} \exp[-\varepsilon_{ij}/\lambda_g]\right)^{\lambda_g}$. This is no longer Type I EV but GEV.
\item The key is the addition of the $\lambda_g$ parameters which govern (roughly) the within group correlation.
\item This distribution is a bit cooked up to get a closed form result, but for $\lambda_g \in [0,1]$ for all $g$ it is consistent with random utility maximization.
\end{itemize}
\end{frame}

\begin{frame}{Nested Logit}
The nested logit choice probabilities are:
\begin{eqnarray*}
P_{ij} = \frac{ e^{V_{ij}/\lambda_g} \left(\sum_{k \in J_g} e^{V_{ik}/\lambda_g} \right)^{\lambda_g -1}}{\sum_{h=1}^G \left(\sum_{k \in J_h} e^{V_{ik}/\lambda_h} \right)^{\lambda_h}}
\end{eqnarray*}
Within the same group $g$ we have IIA and proportional substitution 
\begin{eqnarray*}
\frac{P_{ij}}{P_{ik}} = \frac{ e^{V_{ij}/\lambda_g}}{ e^{V_{ik}/\lambda_g}}
\end{eqnarray*}

But for different groups we do not:
\begin{eqnarray*}
P_{ij} = \frac{ e^{V_{ij}/\lambda_g} \left(\sum_{k \in J_g} e^{V_{ik}/\lambda_g} \right)^{\lambda_g -1}}{ e^{V_{ik}/\lambda_h} \left(\sum_{k \in J_h} e^{V_{ik}/\lambda_h} \right)^{\lambda_h -1}}
\end{eqnarray*}
\end{frame}


\begin{frame}{Nested Logit}
We can take the probabilities and re-write them slightly with the substitution that 
$\lambda_g \cdot \underbrace{\log \left(\sum_{k \in J_g} e^{V_{ik}} \right)}_{IV_{ig}}$.
\begin{eqnarray*}
P_{ij} &=& \frac{ e^{V_{ij}/\lambda_g}}{ \left(\sum_{k \in J_g} e^{V_{ik}/\lambda_g} \right)}
\cdot
\frac{ \left(\sum_{k \in J_g} e^{V_{ik}/\lambda_g} \right)^{\lambda_g}}{\sum_{h=1}^G \left(\sum_{k \in J_h} e^{V_{ik}/\lambda_h} \right)^{\lambda_h}} \\
&=& \underbrace{\frac{ e^{V_{ij}/\lambda_g}}{ \left(\sum_{k \in J_g} e^{V_{ik}/\lambda_g} \right)}}_{P_{i j | g}}
\cdot
\underbrace{\frac{e^{\lambda_g IV_{ig}}}{\sum_{h=1}^{G} e^{\lambda_h IV_{ih}} }}_{P_{ig}}
\end{eqnarray*}
This is the decomposition into two logits that leads to the ``sequential logit'' story.
\end{frame}

\begin{frame}{Nested Logit : Notes}
\begin{itemize}
\item $\lambda_g=1$ is the simple logit case (IIA)
\item $\lambda_g \rightarrow 0$ implies that all consumers stay within the nest.
\item $\lambda < 0$ or $\lambda > 1$ can happen and usually means something is wrong. These models are not generally consistent with RUM. (If you report one in your paper I will reject it).
\item $\lambda$ is often interpreted as a correlation parameter and this is almost true but not exactly!
\item There are other extensions: overlapping nests, or three level nested logit. 
\item In general the hard part is understanding what the appropriate nesting structure is ex ante. Maybe for some problems this is obvious but for many not.
\end{itemize}
\end{frame}

\begin{frame}{Nested Logit : Interpretation}
\begin{itemize}
\item It is convenient to think about the ``sequential choice'' version of the nested logit.
\item In practice it is more accurate to think about the structure it imposes on the correlation of $\varepsilon_i$. 
\item We specify a blocked structure (one block for each nest) and estimate a within vs. across nest correlation parameter.
\end{itemize}
\end{frame}


\begin{frame}{Mixed/ Random Coefficients Logit}
As an alternative, we could have specified an error components structure on $\varepsilon_i$.
\begin{eqnarray*}
U_{ij} = \beta x_{ij} + \underbrace{\nu_i z_{ij} + \varepsilon_{ij}}_{\tilde{\varepsilon}_{ij}}
\end{eqnarray*}
\begin{itemize}
\item The key is that $\nu_i$ is unobserved and mean zero. But that $x_{ij},z_{ij}$ are observed per usual and $\varepsilon_{ij}$ is IID Type I EV.
\item This allows for a heteroskedastic structure on $\varepsilon_{i}$, but only one which we can project down onto the space of $z$.
\end{itemize}
An alternative is to allow for individuals to have random variation in $\beta_i$:
\begin{eqnarray*}
U_{ij} = \beta_i x_{ij} +  \varepsilon_{ij}
\end{eqnarray*}
Which is the random coefficients formulation (these are the same model).
\end{frame}

\begin{frame}{Mixed/ Random Coefficients Logit}
For each individual $i$, the resulting choice probability follows a logit:
\begin{eqnarray*}
P_{ij} = \int \frac{ e^{V_{ij}(\beta_i)}}{\sum_k e^{V_{ik}(\beta_i)}} f(\beta_i | \theta) \partial \beta
\end{eqnarray*}
This structure is quite general:
\begin{itemize}
\item The choice probabilities are know a function of unknown parameters $\theta$.
\item We can allow for there to be two types of $\beta_i$ in the population (high-type, low-type). \alert{latent class model}.
\item We can allow $\beta_i$ to follow an independent normal distribution for each component of $x_{ij}$ such as $\beta_i = \overline{\beta} + \nu_i \sigma$.
\item We can allow for correlated normal draws using the Cholesky root of the covariance matrix.
\item Can allow for non-normal distributions too (lognormal, exponential). Why is normal so easy?
\end{itemize}
\end{frame}

\begin{frame}{Mixed/ Random Coefficients Logit}
\begin{itemize}
\item The structure is extremely flexible but at a cost.
\item We generally must perform the integration numerically.
\item High-dimensional numerical integration is difficult. In fact, integration in dimension 8 or higher makes me very nervous.
\item We need to be parsimonious in how many variables have unobservable heterogeneity.
\item Again observed heterogeneity does not make life difficult so the more of that the better!
\end{itemize}
\end{frame}

\begin{frame}{Mixed/ Random Coefficients Logit}
How do we approximate:
\begin{eqnarray*}
P_{ij} = \int \frac{ e^{V_{ij}(\beta_i)}}{\sum_k e^{V_{ik}(\beta_i)}} f(\beta_i | \theta) \partial \beta
\end{eqnarray*}

\begin{itemize}
\item Monte Carlo Integration
\begin{itemize}
\item Draw $\beta_i$ from the candidate distribution. $[\beta_i^{(1)}, \beta_i^{(2)}, \ldots\beta_i^{(s)}] | \theta$.
\item For each $\beta_i$ calculate $P_{ij}(\beta_i)$.
\item $\frac{1}{S} \sum_{s=1}^S P_{ij} = \widehat{P_{j}^{s}}$
\end{itemize}
\end{itemize}
The way we usually get correlated normal variables (or any normal variables) is to transform independent normals appropriately.
\end{frame}

\begin{frame}{Mixed/ Random Coefficients Logit}
Suppose there is only one random coefficient, and the others are fixed:
\begin{itemize}
\item $f(\beta_i \theta) \sim N(\overline{\beta},\sigma)$.
\item We can re-write this as the integral over a transformed standard normal density
\begin{eqnarray*}
P_{ij}(\theta) = \int \frac{ e^{V_{ij}(\nu_i,\theta)}}{\sum_k e^{V_{ik}(\nu_i,\theta)}} f(\nu_i) \partial \nu
\end{eqnarray*}
\item Monte Carlo Integration: Independent Normal Case
\begin{itemize}
\item Draw $\nu_i$ from the standard normal distribution.
\item Now we can rewrite $\beta_i = \overline{\beta} + \nu_i \sigma$
\item For each $\beta_i$ calculate $P_{ij}(\beta_i)$.
\item $\frac{1}{S} \sum_{s=1}^S P_{ij} = \widehat{P_{j}^{s}}$
\end{itemize}
\item Gaussian Quadrature
\begin{itemize}
\item Or we can draw a non-random set of points $\nu_i$ and corresponding weights $w_i$ and approximate the integral to a high level of polynomial accuracy.
\end{itemize}
\end{itemize}
\end{frame}

\begin{frame}{Quadrature in higher dimensions}
\begin{itemize}
\item Quadrature is great in low dimensions -- but scales badly in high dimensions.
\item If we need $N_a$ points to accurately approximate the integral in $d=1$ then we need $N_a^d$ points in dimension $d$ (using the tensor product of quadrature rules).
\item There is some research on quadrature rules that nest and also how to carefully eliminate points so that the number doesn't grow so quickly.
\item Try \url{sparse-grids.de}
\end{itemize}
\end{frame}

\begin{frame}{Estimation}
How do we actually estimate these models?
\begin{itemize}
\item In practice we should be able to do MLE.
\begin{eqnarray*}
\max_{\theta} \sum_{i=1}^N y_{ij} \log P_{ij}(\theta)
\end{eqnarray*}
\item When we are doing IIA logit, this problem is globally convex and is easy to estimate using Newton's Method.
\item When doing nested logit or random coefficients logit, it generally is non-convex which can make life difficult.
\item The tough part is generally working out what $\frac{\partial \log P_{ij}}{\partial \theta}$ is, especially when we need to simulate to obtain $P_{ij}$.
\item It turns out that MSLE actually has consistent problems for fixed $S$. Why?
\item Alternative? MSM/MoM type estimators (next time).
\end{itemize}
\end{frame}

\begin{frame}{Semi-parametric Alternative}
Fox Kim Bajari Ryan (QE 2011) propose a nice alternative:
\begin{eqnarray*}
g_j(x_i,\beta^r) = \frac{ \exp(x_{ij} \beta^r)}{1+ \sum_{j=1}^J \exp(x_{ij} \beta^r)}\\
\theta^r  \geq 0
\sum_{r=1}^R \theta^r =1\\
E[Y_{ij} - \sum_{r=1}^R \theta^r g_j(x_i,\beta^r)]=0
\end{eqnarray*}
Via constrained LLS.
\end{frame}



\section{Convexity}
\frame{\frametitle{Convexity}
\begin{block}{An optimization problem is convex if}
\begin{eqnarray*}
\min_{x} f(\symbf{x}) &s.t.& h(\symbf{x}) \leq 0 \quad A \symbf{x} = 0
\end{eqnarray*}
\vskip -2ex
\begin{itemize}
\item $f(\symbf{x}),h(\symbf{x)}$ are convex (PSD second derivative matrix)
\item Equality Constraint is affine
\end{itemize}
\end{block}
\begin{exampleblock}{Some helpful identities about convexity}
\begin{itemize}
\item Compositions  and sums of convex functions are convex.
\item Norms $||$ are convex, $\max$ is convex, $\log$ is convex
\item  $ \log(\sum_{i=1}^n \exp(x_i))$ is convex.
\item Fixed Points can introduce non-convexities.
\item Globally convex problems have a unique optimum
\end{itemize}
\end{exampleblock}
}

\frame{\frametitle{Properties of Convex Optimization}
\begin{itemize}
\item If a program is globally convex then it has a unique minimizer that will be found by convex optimizers.
\item If a program is not globally convex, but is convex over a region of the parameter space, then most convex optimization routines find any local minima in the convex hull 
\item Convex optimization routines are unlikely to find local minima (including the global minimum) if they do not begin in the same convex hull as the optimum (starting values matter!).
\item Most good commercial routines are clever about dealing with multiple starting values and handling problems that are well approximated by convex functions.
\item Good Routines use information about sparseness of Hessian -- this generally determines speed.
\end{itemize}
}

\subsection{Nested Logit Example}


%\subsection{Logit and Nested Logit}
%\frame{\frametitle{Logit Model}
%Easy to see that FIML estimator is convex:
%\begin{eqnarray*}
%\min_{\theta} \sum_j q_j \ln \left(\frac{\exp[x_j \beta ]}{1+\sum_j \exp[x_j \beta]} \right)\\
%\min_{\theta} \sum_j q_j \left( x_j \beta  - \ln \left(1+\sum_j \exp[x_j \beta ]\right) \right)\\
%\end{eqnarray*}
%}

\frame{\frametitle{Nested Logit Model}
\begin{exampleblock}{FIML Nested Logit Model is Non-Convex}
\begin{eqnarray*}
\min_{\theta} \sum_j q_j \ln P_j(\theta) \quad \mbox{s.t.} \quad  P_j(\theta) = \frac{e^{x_j \beta/ \lambda}( \sum_{k \in g_l} e^{x_j \beta/ \lambda})^{\lambda-1}}{\sum_{\forall l'} ( \sum_{k \in g_l'} e^{x_j \beta/ \lambda})^{\lambda} }
\end{eqnarray*}
This is a pain to show but the problem is with the cross term $\frac{\partial^2 P_j}{\partial \beta \partial \lambda}$ because $\exp[x_j \beta / \lambda]$ is not convex.
\end{exampleblock}
\begin{exampleblock}{A Simple Substitution Saves the Day:  let $\gamma = \beta / \lambda$}
\begin{eqnarray*}
\min_{\theta} \sum_j q_j \ln P_j(\theta) \quad \mbox{s.t.} \quad  
P_j(\theta) = \frac{e^{x_j \gamma}( \sum_{k \in g_l} e^{x_j \gamma})^{\lambda-1}}{\sum_{\forall l'} ( \sum_{k \in g_l'} e^{x_j \gamma})^{\lambda} }
\end{eqnarray*}
This is much better behaved and easier to optimize.
\end{exampleblock}
}

\frame{\frametitle{Nested Logit Model}
\begin{center}
\rowcolors[]{1}{RoyalBlue!10}{RoyalBlue!20} 
\begin{tabular}{lrrr}
&\bf{Original}\footnote{KNITRO-AMPL} & \bf{Substitution}\footnote{KNITRO-AMPL} & \bf{No Derivatives}\footnote{fminunc-MATLAB} \\
Parameters &49& 49&49\\
Nonlinear $\lambda$ &5 &5 &5\\
Likelihood & 2.279448 &2.279448  & 2.27972\\
Iterations &197 &146 & 352\\
Time & 59.0 s & 10.7 s & 192s  \\
\end{tabular}
\end{center}
Discuss Nelder-Meade
}

\frame{\frametitle{Computing Derivatives}
A key aspect of any optimization problem is going to be computing the derivatives (first and second) of the model.  There are some different approaches
\begin{itemize}
\item Numerical: Often inaccurate and error prone (why?)
\item Pencil and Paper: this tends to be mistake prone -- but often actually the fastest
\item Automatic (AMPL): Software brute forces through a chain rule calculation at every step (limited language).
\item Symbolic (Maple/Mathematica): software ``knows'' derivatives of certain objects and can do its own simplification.  (limited language).
\end{itemize}
}




\end{document}
