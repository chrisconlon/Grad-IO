\documentclass[aspectratio=169,10pt]{beamer}
\usepackage{teaching_slides}

\title{Antitrust}
\author{Chris Conlon}
\institute{Grad IO}
\date{\today}

\begin{document}
\frame{\titlepage}

\begin{frame}
\frametitle{What is Antitrust?}
 \begin{itemize}
\item Historical Debate: should we maximize \alert{efficiency/social surplus} or should we focus on \alert{consumer surplus}
\item Today: Maybe something else like political power? big-ness? profits of small firms?
\begin{itemize}
\item US antitrust law focuses primarily on harm to consumers.
\item EC tends to also worry about harm to competing firms.
\item Canada had something like ``total surplus'' as objective until recently. 
 \end{itemize}
 \item We know about DWL from market power from undergrad economics. However, without profits, why would firms innovate or perform R\&D?
 \begin{itemize}
\item Law understands this and awards temporarily monopolies via patents.
 \end{itemize}
\item Today, I am going to focus mostly on \alert{horizontal mergers} among competitors.
\begin{itemize}
\item Most of this is known as \alert{unilateral effects} (which is a terrible name).
\item Also worry about \alert{coordinated effects} which mean the nature of equilibrium changes.
 \end{itemize}
 \end{itemize}
\end{frame}

\begin{frame}
\frametitle{Antitrust Legislation : Sherman Act (1890)}
 \begin{description}
\item [Section 1]``Every contract, combination in the form of trust or otherwise, or conspiracy, in restraint of trade or commerce among the several States, or with foreign nations, is declared to be illegal'' (Violation involves an \alert{agreement}).
\item [Section 2] ``Every person who shall monopolize, or attempt to monopolize, or combine or conspire with any other person or persons, to monopolize any part of the trade or commerce among the several States, or with foreign nations, shall be deemed guilty of a felony''.
 \end{description}
 Three \textit{per se} violations
 \begin{itemize}
 \item (1) price fixing (2) horizontal market division (3) refusals to deal.
 \item Other violations are \textit{rule of reason} (mostly).
 \end{itemize}
 
\end{frame}

\begin{frame}
\frametitle{Antitrust Legislation : Clayton Act (1914)}
 \begin{description}
\item [Section 2] Prohibits some forms of price discrimination, but only when it lessens competition.
\item [Section 3] Prohibits sales based on the condition that the buyer not buy from your competitor (includes tying and exclusive dealing), but only when effect may be to substantially lessen competition.
\item [Section 7] Prohibits mergers where the effect of such acquisition may be substantially to lessen competition, or tend to create a monopoly in any line of commerce.
\item [Section 8] Prevents a person from being a director of multiple competing firms.
 \end{description}
\end{frame}

\begin{frame}
\frametitle{Antitrust Legislation : Hart-Scott-Rodino Act (1976)}
 \begin{itemize}
\item Required pre-notification and registration of large mergers
\begin{itemize}
\item Transaction: \$78.2 million
\item Size of Person: \$156.3 M with target of \$15.6 M or total transaction of \$312.6M
\item These are ``inflation adjusted'' each year.
\end{itemize}
\item Initial review period is 30 days after which DOJ/FTC can request additional information or allow merger to proceed.
\item Second review usually involves detailed information about  price-cost margins, market shares, etc. (Usually more info available than to academic researchers).
\item Can request information company would reasonably have (customer surveys, etc.).
\item After second review can ask for \alert{injunctive relief} or \alert{remedies} which merging parties can oppose in court.
 \end{itemize}
\end{frame}

\begin{frame}{Wollman: AER: Insights: Stealth Consolidation}
Abrupt change to the transaction size was passed with other legislation, led to large change in newly exempted merger filings (for horizontal mergers).
\begin{center}
\includegraphics[height=0.8\textheight]{resources/wollman_consolidation.png}
\end{center}
\end{frame}

\begin{frame}
\frametitle{DOJ/FTC Horizontal Merger Guidelines}
 \begin{itemize}
\item DOJ/FTC describe markets as:
\begin{itemize}
\item Highly Concentrated: $HHI \geq 2500$.
\item Moderately Concentrated: $HHI \in [1500,2500]$. $\Delta HHI \geq 250$ merits scrutiny.
\item Un-Concentrated: $HHI \leq 1500$.
\end{itemize}
\item Also consider \alert{unilateral effects}/UPP and \alert{coordinated effects}.
\item Three steps:
\begin{enumerate}
\item Market Definition
\item Measure Concentration/Initial Screening
\item Merger Simulation
\end{enumerate}
 \end{itemize}
\end{frame}

\begin{frame}
\frametitle{Step 1: Market Definition}
SSNIP
 \begin{itemize}
\item Small but significant and non-transitory increase in price (SSNIP): smallest relevant market where a hypothetical monopolist could impose a 5\% price increase. (For at least one year).
\item Under linear demand this amounts to a price cost margin and an elasticity (sometimes the \alert{critical elasticity}).
 \end{itemize}
 Tricky Examples:
  \begin{itemize}
\item \textit{FTC vs. Whole Foods/Wild Oats}
\item Cellophane Fallacy (\textit{U.S. v. DuPont (1956)})
 \end{itemize}
\end{frame}

\begin{frame}{SSNIP/HMT Example}
Courts often rely on the \alert{hypothetical monopolist test} or \alert{SSNIP} to define a market:
\begin{enumerate}
\item Start with a candidate market\\
Suppose you’re defining the market for Coca-Cola.

\item  If a monopolist controlled Coca-Cola only, could it raise price by 5–10\% without losing too many customers?
\begin{itemize}
\item If customers switch heavily to Pepsi, then the SSNIP is not profitable.
\item So, expand the market to include Pepsi.
\end{itemize}
\item Re-test. Now suppose the market is Coca-Cola + Pepsi.
\item Ask again: If a monopolist controlled both, could they raise price 5–10\%?
\begin{itemize}
\item If consumers switch to Sprite, bottled water, iced tea, etc., such that the price increase would fail, expand again.
\end{itemize}
\item Continue until the hypothetical monopolist could profitably raise prices — that defines the relevant market.
\end{enumerate}
\end{frame}

\begin{frame}{Problems with HMT}
There are (many) problems with HMT
\begin{itemize}
\item To avoid the cellophane fallacy, we'd like to ask how substitutable products are not at \alert{observed market prices} but rather at \alert{competitive prices} (which we don't see/know).
\item There are likely many such markets that satisfy the SSNIP test, and it seems like the order in which we add products will likely matter. The ``greedy algorithm'' was described above.
\begin{itemize}
    \item After stopping we might apply \alert{smallest market principle} and see if we could still sustain an SSNIP after eliminating one or more products from the market.
    \item In reality, Coca-Cola owns many products (Coke, Sprite, Diet Coke, Powerade) already, and maybe we should already take that into account? (Do we add firms or products to the HMT?)
    \item Do I add products that are closer substitutes to $A$ or products that increase the average price in the market by the most? (ie: closer to substitute $B$).
\end{itemize}
\item You and I may propose candidate markets that: (a) satisfy SSNIP test; (b) satisfy smallest market principle and (c) don't overlap much at all.
\end{itemize}
\end{frame}

\begin{frame}{Critical Loss}
A similar idea is \alert{critical loss}
\begin{align*}
    CL = \frac{\Delta p}{\Delta p +L} \text{ where  } L=\frac{p-mc}{p}
\end{align*}
Example $L = 40\%$ and $\Delta p=.05$ (or 5\% increase)
\begin{align*}
    CL = \frac{.05 }{.05 + 0.4} = 0.111 \text { or 11\% sales decline }
\end{align*}
\begin{itemize}
    \item We're just describing an (own) elasticity! If actual loss (or elasticity) exceeds $CL$ than price increase isn't profitable    
    \item But this is kind of just coming from a single product lerner index (!)
\end{itemize}
\end{frame}



\begin{frame}{Aggregate Diversion}
A related idea is \alert{aggregate diversion ratio}
\begin{align*}
D_{j \rightarrow 0}^{\mathcal{M}} =  \frac{\sum_{k \in \mathcal{M}} \frac{\partial\, q_k}{\partial\, p_j}}{\frac{\partial\, q_k}{\partial\, p_j}}
\end{align*}
\begin{itemize}
    \item Raise the price $p_j$ and calculate the diversion ratio to the outside good (all products outside the candidate market $\mathcal{M}$).
    \item You only want to increase prices if $\frac{L}{ \Delta p} > D_{j \rightarrow 0}^{\mathcal{M}}$
    \item High outside-good diversion means everyone leaves the market (market too small).
    \item Low outside-good diversion means you recapture consumers.
    \item Note: all of these ideas impose a lot of symmetry about price cost margins!
\end{itemize}
\end{frame}

\begin{frame}{Market Definition}
A necessary or \alert{unnecessary} evil?
\begin{itemize}
    \item If you had sufficient data to estimate cross-elasticities or diversion ratios, it is probably easier to tell us if \alert{merger will increase prices} than whether a hypothetical monopolist would raise prices 5\% above the perfect competition level given a very different ownership structrure than the current market.
    \item But judges/courts insist on ``it will raise prices? but in which \textit{market}?''
    \item Personal View: if a merger of A\&B is likely to increase prices by more than some amount (5\%) then we already know: (a) they are in the same market and (b) merger will substially lessen competiion
\end{itemize}
Shows up not only in \alert{merger cases} but in \alert{monopolization cases} as well.
\end{frame}




\begin{frame}
\frametitle{Step 2: Concentration/Screening}
 \begin{itemize}
\item After we define the relevant market, compute the relevant HHI or UPP.
\item There can be both geographic and product market issues in the relevant market.
\item Some markets may be highly concentrated and others may not be.
\item Can ask for \alert{divestitures} as part of a \alert{remedy} if there are a few problematic markets in an otherwise uncontroversial merger.
 \end{itemize}
\end{frame}


\begin{frame}
\frametitle{Step 3: Merger Simulation}
 \begin{itemize}
\item Simulate the price effects of the merger
\item Take into account likely cost synergies (sometimes there are none).
\item Estimate post-merger prices and welfare.
 \end{itemize}
 This is what we will talk about next.
\end{frame}













\end{document}