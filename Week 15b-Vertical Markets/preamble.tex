% Wrapper to use the canonical shared preamble
% Keep original class options for this folder if needed
\def\beamerclassoptions{[xcolor=pdftex,dvipsnames,table,mathserif,aspectratio=169]}
% -----------------------------------------------------------------------------
% Thin wrapper preamble for Grad-IO slides
% Usage:
%  - Optionally set \def\beamerclassoptions{[<opts>]} before \input-ing this file
%    so a slide can control per-file beamer options (e.g. [handout,aspectratio=169]).
%  - This file ONLY issues the \documentclass once (honoring \beamerclassoptions)
%    and then loads the shared package `gradio-preamble.sty` which contains the
%    guarded package loads and macro definitions.
% -----------------------------------------------------------------------------
\ifdefined\beamerclassoptions
        % Expand \beamerclassoptions (which should be like "[notes=show]") safely
        \begingroup\edef\x{\endgroup\noexpand\documentclass\beamerclassoptions{beamer}}\x
\else
        \documentclass[handout,10pt,aspectratio=169]{beamer}
\fi

% Load the canonical shared preamble package. Try the local resources folder first
% so slide-level inputs continue to work in-place; if installed system-wide then
% \RequirePackage will also find it by name.
\IfFileExists{./gradio-preamble.sty}{\RequirePackage{./gradio-preamble}}{%
    \IfFileExists{gradio-preamble.sty}{\RequirePackage{gradio-preamble}}{%
        \RequirePackage{gradio-preamble}% final fallback to system-installed package
    }
}

% Any slide-specific packages or macros should be declared in the slide file
% after `% -----------------------------------------------------------------------------
% Thin wrapper preamble for Grad-IO slides
% Usage:
%  - Optionally set \def\beamerclassoptions{[<opts>]} before \input-ing this file
%    so a slide can control per-file beamer options (e.g. [handout,aspectratio=169]).
%  - This file ONLY issues the \documentclass once (honoring \beamerclassoptions)
%    and then loads the shared package `gradio-preamble.sty` which contains the
%    guarded package loads and macro definitions.
% -----------------------------------------------------------------------------
\ifdefined\beamerclassoptions
        % Expand \beamerclassoptions (which should be like "[notes=show]") safely
        \begingroup\edef\x{\endgroup\noexpand\documentclass\beamerclassoptions{beamer}}\x
\else
        \documentclass[handout,10pt,aspectratio=169]{beamer}
\fi

% Load the canonical shared preamble package. Try the local resources folder first
% so slide-level inputs continue to work in-place; if installed system-wide then
% \RequirePackage will also find it by name.
\IfFileExists{./gradio-preamble.sty}{\RequirePackage{./gradio-preamble}}{%
    \IfFileExists{gradio-preamble.sty}{\RequirePackage{gradio-preamble}}{%
        \RequirePackage{gradio-preamble}% final fallback to system-installed package
    }
}

% Any slide-specific packages or macros should be declared in the slide file
% after `% -----------------------------------------------------------------------------
% Thin wrapper preamble for Grad-IO slides
% Usage:
%  - Optionally set \def\beamerclassoptions{[<opts>]} before \input-ing this file
%    so a slide can control per-file beamer options (e.g. [handout,aspectratio=169]).
%  - This file ONLY issues the \documentclass once (honoring \beamerclassoptions)
%    and then loads the shared package `gradio-preamble.sty` which contains the
%    guarded package loads and macro definitions.
% -----------------------------------------------------------------------------
\ifdefined\beamerclassoptions
        % Expand \beamerclassoptions (which should be like "[notes=show]") safely
        \begingroup\edef\x{\endgroup\noexpand\documentclass\beamerclassoptions{beamer}}\x
\else
        \documentclass[handout,10pt,aspectratio=169]{beamer}
\fi

% Load the canonical shared preamble package. Try the local resources folder first
% so slide-level inputs continue to work in-place; if installed system-wide then
% \RequirePackage will also find it by name.
\IfFileExists{./gradio-preamble.sty}{\RequirePackage{./gradio-preamble}}{%
    \IfFileExists{gradio-preamble.sty}{\RequirePackage{gradio-preamble}}{%
        \RequirePackage{gradio-preamble}% final fallback to system-installed package
    }
}

% Any slide-specific packages or macros should be declared in the slide file
% after `\input{.../resources/preamble.tex}`.

`.

`.



% packages
%\usepackage{newpxmath} % math font is Palatino compatible
%\usepackage[nomath]{fontspec}

\usepackage{setspace}
\usepackage{xcolor}
\usepackage{soul} % for \st
\usepackage{hyperref} % for links
\definecolor{links}{HTML}{2A1B81}
\hypersetup{colorlinks,linkcolor=,urlcolor=links}


% table stuff
\usepackage{chronosys}
\usepackage{verbatim}
% \pagenumbering{arabic}
\usepackage{tabularx}
\usepackage{booktabs}
\usepackage{ragged2e}
\usepackage{mathtools}

% R Code
\usepackage{listings}
\usepackage{courier}
\lstset{basicstyle=\scriptsize\ttfamily,breaklines=true}
\lstset{framextopmargin=50pt,frame=bottomline}

% themes
\usetheme[progressbar=frametitle, block=fill]{metropolis}
\useoutertheme{metropolis}
\useinnertheme{metropolis}

% colors
\definecolor{dimwhite}{rgb}{0.99, 0.99, 0.99}
\definecolor{charcoal}{rgb}{0.21, 0.27, 0.31}
\definecolor{slategray}{rgb}{0.44, 0.5, 0.56}
\definecolor{dimgray}{rgb}{0.41, 0.41, 0.41}
\definecolor{bleudefrance}{rgb}{0.19, 0.55, 0.91}

% beamer options
\setbeamercolor{author}{fg=charcoal}
\setbeamercolor{background canvas}{bg=white}
\setbeamercolor{section in toc}{fg=charcoal}
\setbeamercolor{subsection in toc}{fg=dimgray}
\setbeamercolor{frametitle}{bg=dimwhite, fg=charcoal}
\setbeamercolor{progress bar}{fg=slategray, bg=fg!50!black!30}
\setbeamercovered{transparent}
\setbeamertemplate{itemize items}[triangle]
\setbeamertemplate{itemize subitem}[circle]
\setbeamertemplate{itemize subsubitem}[square]
\setbeamersize{text margin left=7mm,text margin right=7mm} 

% new commands
\newcommand{\q}[1]{``#1''}
\newcommand{\hs}[1]{\textsc{\hfill\scriptsize\color{dimgray}#1}}
\newcommand{\g}[1]{{\color{gray}#1}}
\newcommand{\dg}[1]{{\color{dimgray}#1}}
\newcommand{\sg}[1]{{\color{slategray}#1}}
\newcommand{\bdf}[1]{{\color{bleudefrance}#1}}
\newcommand{\itemcolor}[1]{\renewcommand{\makelabel}[1]{\color{#1}\hfil ##1}}
\newcommand\Wider[2][2em]{
\makebox[\linewidth][c]{
  \begin{minipage}{\dimexpr\textwidth+#1\relax}
  \raggedright#2
  \end{minipage}
  }
}

% misc
\linespread{1.35}

% Math stuff
\newcommand{\norm}[1]{\left\lVert#1\right\rVert}
\newcommand{\R}{\mathbb{R}}
\newcommand{\E}{\mathbb{E}}
\newcommand{\V}{\mathbb{V}}
\newcommand{\probP}{\mathbb{P}}
\newcommand{\ol}{\overline}
%\newcommand{\ul}{\underline}
\newcommand{\pp}{{\prime \prime}}
\newcommand{\ppp}{{\prime \prime \prime}}
\newcommand{\policy}{\gamma}
\newcommand{\plim}{ \overset{p}{\to}}
\newcommand{\hnot}{ \overset{H_0}{\to}}

% Causal Graphs
\usetikzlibrary{shapes,decorations,arrows,calc,arrows.meta,fit,positioning}
\tikzset{
    -Latex,auto,node distance =1 cm and 1 cm,semithick,
    state/.style ={ellipse, draw, minimum width = 0.7 cm},
    point/.style = {circle, draw, inner sep=0.04cm,fill,node contents={}},
    bidirected/.style={Latex-Latex,dashed},
    el/.style = {inner sep=2pt, align=left, sloped}
}