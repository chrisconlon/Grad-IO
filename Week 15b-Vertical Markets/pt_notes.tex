% Wrapper to use the canonical shared preamble
% -----------------------------------------------------------------------------
% Thin wrapper preamble for Grad-IO slides
% Usage:
%  - Optionally set \def\beamerclassoptions{[<opts>]} before \input-ing this file
%    so a slide can control per-file beamer options (e.g. [handout,aspectratio=169]).
%  - This file ONLY issues the \documentclass once (honoring \beamerclassoptions)
%    and then loads the shared package `gradio-preamble.sty` which contains the
%    guarded package loads and macro definitions.
% -----------------------------------------------------------------------------
\ifdefined\beamerclassoptions
        % Expand \beamerclassoptions (which should be like "[notes=show]") safely
        \begingroup\edef\x{\endgroup\noexpand\documentclass\beamerclassoptions{beamer}}\x
\else
        \documentclass[handout,10pt,aspectratio=169]{beamer}
\fi

% Load the canonical shared preamble package. Try the local resources folder first
% so slide-level inputs continue to work in-place; if installed system-wide then
% \RequirePackage will also find it by name.
\IfFileExists{./gradio-preamble.sty}{\RequirePackage{./gradio-preamble}}{%
    \IfFileExists{gradio-preamble.sty}{\RequirePackage{gradio-preamble}}{%
        \RequirePackage{gradio-preamble}% final fallback to system-installed package
    }
}

% Any slide-specific packages or macros should be declared in the slide file
% after `% -----------------------------------------------------------------------------
% Thin wrapper preamble for Grad-IO slides
% Usage:
%  - Optionally set \def\beamerclassoptions{[<opts>]} before \input-ing this file
%    so a slide can control per-file beamer options (e.g. [handout,aspectratio=169]).
%  - This file ONLY issues the \documentclass once (honoring \beamerclassoptions)
%    and then loads the shared package `gradio-preamble.sty` which contains the
%    guarded package loads and macro definitions.
% -----------------------------------------------------------------------------
\ifdefined\beamerclassoptions
        % Expand \beamerclassoptions (which should be like "[notes=show]") safely
        \begingroup\edef\x{\endgroup\noexpand\documentclass\beamerclassoptions{beamer}}\x
\else
        \documentclass[handout,10pt,aspectratio=169]{beamer}
\fi

% Load the canonical shared preamble package. Try the local resources folder first
% so slide-level inputs continue to work in-place; if installed system-wide then
% \RequirePackage will also find it by name.
\IfFileExists{./gradio-preamble.sty}{\RequirePackage{./gradio-preamble}}{%
    \IfFileExists{gradio-preamble.sty}{\RequirePackage{gradio-preamble}}{%
        \RequirePackage{gradio-preamble}% final fallback to system-installed package
    }
}

% Any slide-specific packages or macros should be declared in the slide file
% after `\input{.../resources/preamble.tex}`.

`.



% silence some Metropolis warnings
\usepackage{silence}
\WarningFilter{beamerthememetropolis}{You need to compile with XeLaTeX or LuaLaTeX}
\WarningFilter{latexfont}{Font shape}
\WarningFilter{latexfont}{Some font}

% define custom colors
\definecolor{dark gray}{HTML}{444444}
\definecolor{light gray}{HTML}{777777}
\definecolor{dark red}{HTML}{BB0000}
\definecolor{dark green}{HTML}{00BB00}

% configure metropolis
\usetheme[numbering=fraction]{metropolis}
\setbeamercolor{background canvas}{bg=white}
\setbeamercolor{frametitle}{bg=dark gray}
\setbeamercolor{alerted text}{fg=dark red}
\setbeamercolor{item projected}{bg=dark red}
\setbeamercolor{local structure}{fg=dark red}
\setbeamersize{text margin left=0.5cm,text margin right=0.5cm}
\setbeamercovered{transparent=10}

% use thicker lines
\makeatletter
\setlength{\metropolis@titleseparator@linewidth}{1pt}
\setlength{\metropolis@progressonsectionpage@linewidth}{1pt}
\makeatother

% custom bullet points

% Minimal wrapper to use the canonical shared preamble for the Conduct lectures.
\def\beamerclassoptions{[10pt,aspectratio=169]}
% -----------------------------------------------------------------------------
% Thin wrapper preamble for Grad-IO slides
% Usage:
%  - Optionally set \def\beamerclassoptions{[<opts>]} before \input-ing this file
%    so a slide can control per-file beamer options (e.g. [handout,aspectratio=169]).
%  - This file ONLY issues the \documentclass once (honoring \beamerclassoptions)
%    and then loads the shared package `gradio-preamble.sty` which contains the
%    guarded package loads and macro definitions.
% -----------------------------------------------------------------------------
\ifdefined\beamerclassoptions
        % Expand \beamerclassoptions (which should be like "[notes=show]") safely
        \begingroup\edef\x{\endgroup\noexpand\documentclass\beamerclassoptions{beamer}}\x
\else
        \documentclass[handout,10pt,aspectratio=169]{beamer}
\fi

% Load the canonical shared preamble package. Try the local resources folder first
% so slide-level inputs continue to work in-place; if installed system-wide then
% \RequirePackage will also find it by name.
\IfFileExists{./gradio-preamble.sty}{\RequirePackage{./gradio-preamble}}{%
    \IfFileExists{gradio-preamble.sty}{\RequirePackage{gradio-preamble}}{%
        \RequirePackage{gradio-preamble}% final fallback to system-installed package
    }
}

% Any slide-specific packages or macros should be declared in the slide file
% after `% -----------------------------------------------------------------------------
% Thin wrapper preamble for Grad-IO slides
% Usage:
%  - Optionally set \def\beamerclassoptions{[<opts>]} before \input-ing this file
%    so a slide can control per-file beamer options (e.g. [handout,aspectratio=169]).
%  - This file ONLY issues the \documentclass once (honoring \beamerclassoptions)
%    and then loads the shared package `gradio-preamble.sty` which contains the
%    guarded package loads and macro definitions.
% -----------------------------------------------------------------------------
\ifdefined\beamerclassoptions
        % Expand \beamerclassoptions (which should be like "[notes=show]") safely
        \begingroup\edef\x{\endgroup\noexpand\documentclass\beamerclassoptions{beamer}}\x
\else
        \documentclass[handout,10pt,aspectratio=169]{beamer}
\fi

% Load the canonical shared preamble package. Try the local resources folder first
% so slide-level inputs continue to work in-place; if installed system-wide then
% \RequirePackage will also find it by name.
\IfFileExists{./gradio-preamble.sty}{\RequirePackage{./gradio-preamble}}{%
    \IfFileExists{gradio-preamble.sty}{\RequirePackage{gradio-preamble}}{%
        \RequirePackage{gradio-preamble}% final fallback to system-installed package
    }
}

% Any slide-specific packages or macros should be declared in the slide file
% after `\input{.../resources/preamble.tex}`.

`.





\title []{Pass-through}
\author{C.Conlon }
\institute{Internal Notes }
\date{Fall 2022}
\setbeamerfont{equation}{size=\tiny}
\begin{document}

\begin{frame}
\titlepage
\end{frame}




%%%%%%%%%%%%%%%%%%%%%%%%%%%%%%%%%%%%%%%%%%%%%%%%%%
%%%%%%%%%%%%%%%%%%%%%%%%%%%%%%%%%%%%%%%%%%%%%%%%%%%


\begin{frame}{Villas Boas (ReStud 2007)}
Retailer and Wholesaler FOC given by:
\begin{align*}
\symbf{p^r} &= \underbrace{\symbf{p^w} +\symbf{c^r}}_{\symbf{mc^r}} -(\mathcal{H}_r \odot \Delta_{r}(\symbf{p^r}))^{-1} \symbf{s}(\symbf{p^r})\\
\symbf{p^w}  &= \symbf{mc^w} + \left(\mathcal{H}_{w} \odot \left( \frac{\partial \symbf{p^r}}{\partial \symbf{p^w}} \cdot  \Delta_r(\symbf{p^r} ) \right) \right)^{-1} \symbf{s}(\symbf{p^r})
\end{align*}
\begin{itemize}
  \item $\Delta_r$ is matrix of (retail) demand derivatives $\frac{\partial\, \symbf{s}}{\partial\, \symbf{p}}$.
\item $\mathcal{H}_r,\mathcal{H}_w$  ownership matrix $(j,k)=1$ if both products sold by same retailer/wholesaler.
\item $\frac{\partial\, \symbf{p^r}}{\partial\, \symbf{p^w}}$ is the \alert{pass-through matrix} (NEW!)
\end{itemize}
Challenge: We want $\symbf{p^r}(\symbf{p^w})$ and $\symbf{mc^w}$ but we only have implicit solution for retailer FOC.
\end{frame}

\begin{frame}{How do we get pass-through?}
The \alert{pass-through matrix} $\frac{\partial \symbf{p^r}}{\partial \symbf{p^w}}$ can be obtained in one of two ways:
\begin{enumerate}
\item Numerically: perturbing the retailer's marginal costs for each possible choice of $k$ and solving
\begin{align*}
\symbf{p^r} &=\symbf{mc^r} + e_k -(\mathcal{H}_r \odot \Delta_{r}(\symbf{p^r}))^{-1} \symbf{s}(\symbf{p^r})\\
\end{align*}
(Use Morrow Skerlos (2011) formulation and solve for every $(j,k)$ pair).
\item Analytic: Use the retailer's FOC and apply the implicit function theorem.
\begin{align}
\tag{retailer FOC}
 f(\symbf{p^r},\symbf{mc^r}) &\equiv \symbf{p^r}  - \symbf{mc^r}-  \left(\mathcal{H}_{r} \odot \Delta(\symbf{p^r}) \right)^{-1} \symbf{s}(\symbf{p^r})=0 
\end{align}
See Jaffe Weyl (AEJM 2013) or Miller Weinberg (2017 Appendix E) or Conlon Rao (2022).\\
\alert{This is what PyBLP does}.
  \end{enumerate}

\end{frame}

\begin{frame}{Multivariate IFT: Easy Part}
The multivariate IFT says that for some system of $J$ nonlinear equations 
\begin{align*}
f(\symbf{p^r},\symbf{p^w}) \equiv [F_1(\symbf{p^r},\symbf{p^w}), \ldots, F_J(\symbf{p^r},\symbf{p^w})]=[0,\ldots,0]
\end{align*}
with $J$ endogenous variables $\symbf{p^r}$ and $J$ exogenous parameters $\symbf{p^w}$.
\begin{align}
\label{eq:ptr_matrix}
\tag{PTR}
\frac{\partial \symbf{p^r}}{\partial \symbf{p^w}}
=-\left(\begin{array}{ccc}
\frac{\partial F_{1}}{\partial p_{1}^r} & \ldots & \frac{\partial F_{1}}{\partial p_{J}^r} \\
\ldots & \ldots & \ldots \\
\frac{\partial F_{J}}{\partial p_{1}^r} & \ldots & \frac{\partial F_{J}}{\partial p_{J}^r}
\end{array}\right)^{-1} \cdot \underbrace{\left(\begin{array}{l}
\frac{\partial F_{1}}{\partial p_{k}^w} \\
\ldots \\
\frac{\partial F_{J}}{\partial p_{k}^w}
\end{array}\right)}_{= -\mathbb{I}_J}
\end{align}
Because the system of equations is additive in $\symbf{mc^r} = \symbf{c^r} + \symbf{p^w}$ this simplifies dramatically.
\end{frame}


\begin{frame}{Multivariate IFT: Hard Part}
Use the substitution $\Omega(\symbf{p^r}) \equiv \mathcal{H}_r \odot \Delta_{r}(\symbf{p^r})$, and differentiate the wholesalers' system of FOC's with respect to $p_l$, to get the $J \times J$ matrix with columns $l$ given by:
\begin{align}
\frac{\partial f(\symbf{p^r},\symbf{p^w})}{\partial p_l^r} \equiv e_l - \Omega^{-1}(\symbf{p^r})
\left[  \mathcal{H}_{r} \odot \frac{\partial\, \Delta(\symbf{p^r})}{\partial\, p_l^r} \right]
\Omega^{-1}(\symbf{p^r})\,
\symbf{s}(\symbf{p^r}) -\Omega^{-1}(\symbf{p^r})\, \frac{\partial \symbf{s}(\symbf{p^r})}{\partial p_l^r}.
\end{align}
The complicated piece is the demand Hessian: a $J \times J \times J$ tensor with elements $(j,k,l)$, $\frac{\partial^2 s_j}{\partial p_k^r \partial p_l^r} = \frac{\partial^2 \symbf{s}}{\partial \symbf{p^r} \partial p_l^r}=\frac{\partial\, \Delta(\symbf{p^r})}{\partial\, p_l^r}$.\\

This also shows a key relationship between \alert{pass through} and \alert{demand curvature} (2nd derivatives).
\end{frame}



\begin{frame}{Pass-through Counterfactuals?}
\footnotesize
How do we solve for $p^w$ under a counterfactual pass-through matrix?
\begin{itemize}
\item Idea: pass-through only augments the matrix $\Delta_r(\symbf{p^r})$.
\item Example: a constant sales tax rate $P \equiv \frac{\partial \symbf{p^r}}{\partial \symbf{p^w}} = \text{diag}(1+\tau_r)$
\end{itemize}
\begin{align*}
\symbf{p^w}  &= \symbf{mc^w} + \left(\mathcal{H}_{w} \odot \left( \frac{\partial \symbf{p^r}}{\partial \symbf{p^w}} \cdot  \Delta_r(\symbf{p^r} ) \right) \right)^{-1} \symbf{s}(\symbf{p^r})
\end{align*}

Adapt the Morrow Skerlos $\zeta$ fixed point where $P \Delta(\symbf{p}_t) = P \Lambda_t\left(\symbf{p}_t\right)- P \Gamma_t\left(\symbf{p}_t\right)$
\begin{align*}
\symbf{p}_t &\leftrightarrow \symbf{c}_t+\symbf{\zeta}_t\left(\symbf{p}_t\right) \quad \text { where }\\
 \symbf{\zeta}_t\left(\symbf{p}_t\right)&=\Lambda_t\left(\symbf{p}_t\right)^{-1} \alert{P^{-1}}\left[\mathcal{H}_t^* \odot \alert{P}\, \Gamma_t\left(\symbf{p}_t\right)\right]\left(\symbf{p}_t-\symbf{c}_t\right)-\Lambda_t\left(\symbf{p}_t\right)^{-1} \alert{P^{-1}} \symbf{s}_t\left(\symbf{p}_t\right)
\end{align*}
For diagonal $P$ (not sure about general case with Hadamard product):
\begin{align*}
 \symbf{\zeta}_t\left(\symbf{p}_t\right)=\Lambda_t\left(\symbf{p}_t\right)^{-1}\left[\mathcal{H}_t^* \odot  \Gamma_t\left(\symbf{p}_t\right)\right]\left(\symbf{p}_t-\symbf{c}_t\right)-\Lambda_t\left(\symbf{p}_t\right)^{-1} \alert{P^{-1}} \symbf{s}_t\left(\symbf{p}_t\right)
\end{align*}
 
\end{frame}


\end{document}