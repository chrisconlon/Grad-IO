% Unified wrapper preamble that forwards to the central shared preamble.
% This keeps slide styling consistent while allowing individual folders to
% include a local `preamble.tex` that simply points here.
% -----------------------------------------------------------------------------
% Thin wrapper preamble for Grad-IO slides
% Usage:
%  - Optionally set \def\beamerclassoptions{[<opts>]} before \input-ing this file
%    so a slide can control per-file beamer options (e.g. [handout,aspectratio=169]).
%  - This file ONLY issues the \documentclass once (honoring \beamerclassoptions)
%    and then loads the shared package `gradio-preamble.sty` which contains the
%    guarded package loads and macro definitions.
% -----------------------------------------------------------------------------
\ifdefined\beamerclassoptions
        % Expand \beamerclassoptions (which should be like "[notes=show]") safely
        \begingroup\edef\x{\endgroup\noexpand\documentclass\beamerclassoptions{beamer}}\x
\else
        \documentclass[handout,10pt,aspectratio=169]{beamer}
\fi

% Load the canonical shared preamble package. Try the local resources folder first
% so slide-level inputs continue to work in-place; if installed system-wide then
% \RequirePackage will also find it by name.
\IfFileExists{./gradio-preamble.sty}{\RequirePackage{./gradio-preamble}}{%
    \IfFileExists{gradio-preamble.sty}{\RequirePackage{gradio-preamble}}{%
        \RequirePackage{gradio-preamble}% final fallback to system-installed package
    }
}

% Any slide-specific packages or macros should be declared in the slide file
% after `% -----------------------------------------------------------------------------
% Thin wrapper preamble for Grad-IO slides
% Usage:
%  - Optionally set \def\beamerclassoptions{[<opts>]} before \input-ing this file
%    so a slide can control per-file beamer options (e.g. [handout,aspectratio=169]).
%  - This file ONLY issues the \documentclass once (honoring \beamerclassoptions)
%    and then loads the shared package `gradio-preamble.sty` which contains the
%    guarded package loads and macro definitions.
% -----------------------------------------------------------------------------
\ifdefined\beamerclassoptions
        % Expand \beamerclassoptions (which should be like "[notes=show]") safely
        \begingroup\edef\x{\endgroup\noexpand\documentclass\beamerclassoptions{beamer}}\x
\else
        \documentclass[handout,10pt,aspectratio=169]{beamer}
\fi

% Load the canonical shared preamble package. Try the local resources folder first
% so slide-level inputs continue to work in-place; if installed system-wide then
% \RequirePackage will also find it by name.
\IfFileExists{./gradio-preamble.sty}{\RequirePackage{./gradio-preamble}}{%
    \IfFileExists{gradio-preamble.sty}{\RequirePackage{gradio-preamble}}{%
        \RequirePackage{gradio-preamble}% final fallback to system-installed package
    }
}

% Any slide-specific packages or macros should be declared in the slide file
% after `\input{.../resources/preamble.tex}`.

`.




\title []{Pass-through}
\author{C.Conlon }
\institute{Internal Notes }
\date{Fall 2022}
\setbeamerfont{equation}{size=\tiny}
\begin{document}

\begin{frame}
\titlepage
\end{frame}




%%%%%%%%%%%%%%%%%%%%%%%%%%%%%%%%%%%%%%%%%%%%%%%%%%
%%%%%%%%%%%%%%%%%%%%%%%%%%%%%%%%%%%%%%%%%%%%%%%%%%%


\begin{frame}{Villas Boas (ReStud 2007)}
Retailer and Wholesaler FOC given by:
\begin{align*}
\mathbf{p^r} &= \underbrace{\mathbf{p^w} +\mathbf{c^r}}_{\mathbf{mc^r}} -(\mathcal{H}_r \odot \Delta_{r}(\mathbf{p^r}))^{-1} \mathbf{s}(\mathbf{p^r})\\
\mathbf{p^w}  &= \mathbf{mc^w} + \left(\mathcal{H}_{w} \odot \left( \frac{\partial \mathbf{p^r}}{\partial \mathbf{p^w}} \cdot  \Delta_r(\mathbf{p^r} ) \right) \right)^{-1} \mathbf{s}(\mathbf{p^r})
\end{align*}
\begin{itemize}
  \item $\Delta_r$ is matrix of (retail) demand derivatives $\frac{\partial\, \mathbf{s}}{\partial\, \mathbf{p}}$.
\item $\mathcal{H}_r,\mathcal{H}_w$  ownership matrix $(j,k)=1$ if both products sold by same retailer/wholesaler.
\item $\frac{\partial\, \mathbf{p^r}}{\partial\, \mathbf{p^w}}$ is the \alert{pass-through matrix} (NEW!)
\end{itemize}
Challenge: We want $\mathbf{p^r}(\mathbf{p^w})$ and $\mathbf{mc^w}$ but we only have implicit solution for retailer FOC.
\end{frame}

\begin{frame}{How do we get pass-through?}
The \alert{pass-through matrix} $\frac{\partial \mathbf{p^r}}{\partial \mathbf{p^w}}$ can be obtained in one of two ways:
\begin{enumerate}
\item Numerically: perturbing the retailer's marginal costs for each possible choice of $k$ and solving
\begin{align*}
\mathbf{p^r} &=\mathbf{mc^r} + e_k -(\mathcal{H}_r \odot \Delta_{r}(\mathbf{p^r}))^{-1} \mathbf{s}(\mathbf{p^r})\\
\end{align*}
(Use Morrow Skerlos (2011) formulation and solve for every $(j,k)$ pair).
\item Analytic: Use the retailer's FOC and apply the implicit function theorem.
\begin{align}
\tag{retailer FOC}
 f(\mathbf{p^r},\mathbf{mc^r}) &\equiv \mathbf{p^r}  - \mathbf{mc^r}-  \left(\mathcal{H}_{r} \odot \Delta(\mathbf{p^r}) \right)^{-1} \mathbf{s}(\mathbf{p^r})=0 
\end{align}
See Jaffe Weyl (AEJM 2013) or Miller Weinberg (2017 Appendix E) or Conlon Rao (2022).\\
\alert{This is what PyBLP does}.
  \end{enumerate}

\end{frame}

\begin{frame}{Multivariate IFT: Easy Part}
The multivariate IFT says that for some system of $J$ nonlinear equations 
\begin{align*}
f(\mathbf{p^r},\mathbf{p^w}) \equiv [F_1(\mathbf{p^r},\mathbf{p^w}), \ldots, F_J(\mathbf{p^r},\mathbf{p^w})]=[0,\ldots,0]
\end{align*}
with $J$ endogenous variables $\mathbf{p^r}$ and $J$ exogenous parameters $\mathbf{p^w}$.
\begin{align}
\label{eq:ptr_matrix}
\tag{PTR}
\frac{\partial \mathbf{p^r}}{\partial \mathbf{p^w}}
=-\left(\begin{array}{ccc}
\frac{\partial F_{1}}{\partial p_{1}^r} & \ldots & \frac{\partial F_{1}}{\partial p_{J}^r} \\
\ldots & \ldots & \ldots \\
\frac{\partial F_{J}}{\partial p_{1}^r} & \ldots & \frac{\partial F_{J}}{\partial p_{J}^r}
\end{array}\right)^{-1} \cdot \underbrace{\left(\begin{array}{l}
\frac{\partial F_{1}}{\partial p_{k}^w} \\
\ldots \\
\frac{\partial F_{J}}{\partial p_{k}^w}
\end{array}\right)}_{= -\mathbb{I}_J}
\end{align}
Because the system of equations is additive in $\mathbf{mc^r} = \mathbf{c^r} + \mathbf{p^w}$ this simplifies dramatically.
\end{frame}


\begin{frame}{Multivariate IFT: Hard Part}
Use the substitution $\Omega(\mathbf{p^r}) \equiv \mathcal{H}_r \odot \Delta_{r}(\mathbf{p^r})$, and differentiate the wholesalers' system of FOC's with respect to $p_l$, to get the $J \times J$ matrix with columns $l$ given by:
\begin{align}
\frac{\partial f(\mathbf{p^r},\mathbf{p^w})}{\partial p_l^r} \equiv e_l - \Omega^{-1}(\mathbf{p^r})
\left[  \mathcal{H}_{r} \odot \frac{\partial\, \Delta(\mathbf{p^r})}{\partial\, p_l^r} \right]
\Omega^{-1}(\mathbf{p^r})\,
\mathbf{s}(\mathbf{p^r}) -\Omega^{-1}(\mathbf{p^r})\, \frac{\partial \mathbf{s}(\mathbf{p^r})}{\partial p_l^r}.
\end{align}
The complicated piece is the demand Hessian: a $J \times J \times J$ tensor with elements $(j,k,l)$, $\frac{\partial^2 s_j}{\partial p_k^r \partial p_l^r} = \frac{\partial^2 \mathbf{s}}{\partial \mathbf{p^r} \partial p_l^r}=\frac{\partial\, \Delta(\mathbf{p^r})}{\partial\, p_l^r}$.\\

This also shows a key relationship between \alert{pass through} and \alert{demand curvature} (2nd derivatives).
\end{frame}



\begin{frame}{Pass-through Counterfactuals?}
\footnotesize
How do we solve for $p^w$ under a counterfactual pass-through matrix?
\begin{itemize}
\item Idea: pass-through only augments the matrix $\Delta_r(\mathbf{p^r})$.
\item Example: a constant sales tax rate $P \equiv \frac{\partial \mathbf{p^r}}{\partial \mathbf{p^w}} = \text{diag}(1+\tau_r)$
\end{itemize}
\begin{align*}
\mathbf{p^w}  &= \mathbf{mc^w} + \left(\mathcal{H}_{w} \odot \left( \frac{\partial \mathbf{p^r}}{\partial \mathbf{p^w}} \cdot  \Delta_r(\mathbf{p^r} ) \right) \right)^{-1} \mathbf{s}(\mathbf{p^r})
\end{align*}

Adapt the Morrow Skerlos $\zeta$ fixed point where $P \Delta(\boldsymbol{p}_t) = P \Lambda_t\left(\boldsymbol{p}_t\right)- P \Gamma_t\left(\boldsymbol{p}_t\right)$
\begin{align*}
\boldsymbol{p}_t &\leftrightarrow \boldsymbol{c}_t+\boldsymbol{\zeta}_t\left(\boldsymbol{p}_t\right) \quad \text { where }\\
 \boldsymbol{\zeta}_t\left(\boldsymbol{p}_t\right)&=\Lambda_t\left(\boldsymbol{p}_t\right)^{-1} \alert{P^{-1}}\left[\mathcal{H}_t^* \odot \alert{P}\, \Gamma_t\left(\boldsymbol{p}_t\right)\right]\left(\boldsymbol{p}_t-\boldsymbol{c}_t\right)-\Lambda_t\left(\boldsymbol{p}_t\right)^{-1} \alert{P^{-1}} \boldsymbol{s}_t\left(\boldsymbol{p}_t\right)
\end{align*}
For diagonal $P$ (not sure about general case with Hadamard product):
\begin{align*}
 \boldsymbol{\zeta}_t\left(\boldsymbol{p}_t\right)=\Lambda_t\left(\boldsymbol{p}_t\right)^{-1}\left[\mathcal{H}_t^* \odot  \Gamma_t\left(\boldsymbol{p}_t\right)\right]\left(\boldsymbol{p}_t-\boldsymbol{c}_t\right)-\Lambda_t\left(\boldsymbol{p}_t\right)^{-1} \alert{P^{-1}} \boldsymbol{s}_t\left(\boldsymbol{p}_t\right)
\end{align*}
 
\end{frame}


\end{document}