\documentclass[aspectratio=169,11pt]{beamer}
\usepackage{teaching_slides}

\title [Single-agent dynamic optimization models]{Persistent Unobservables}
\author{C.Conlon - Adapted from M. Shum}
\institute{Grad IO}
\date{\today}
\setbeamerfont{equation}{size=\tiny}
\begin{document}


\begin{frame}
\titlepage
\end{frame}


%\begin{frame}
%From Shepard (2015 JMP):
%\begin{itemize}
%\item Wants to measure the impact of ``star hospitals''.
%\item Previous wisdom was that in MA you needed to include Partners Healthcare in your insurer's network.
%\item From 2010-2012 Harvard Pilgrim (\#2) insurer excluded Partners Hospital Network (MGH, Brigham Womens, Harvard-MIT teaching hospitals) from their plan
%\begin{itemize}
%\item Comparable procedures are around 40\% more expensive at Partners' Hospitals
%\item Customers revolted! Employer sponsored left in droves, strikes, etc.
%\item Can't run a network without them.
%\end{itemize}
%\item With the ACA it looks like nobody wants to have Partners in their network anymore.
%\item Consumers face same prices for all hospitals
%\item Idea: two dimensions of heterogeneity. 
%\begin{itemize}
%\item Some people like option of MGH in case they get really sick (rare cancer)
%\item Others go to MGH because they have doctors there, have gone in the past, or enjoy amenities but could get comparable care elsewhere.
%\end{itemize}
%\end{itemize}
%\end{frame}
%
%
%
%\begin{frame}{Persistence: Reduced form proxy}
%From Shepard (2015 JMP):
%\begin{center}
%\includegraphics[width=4.5in]{resources/shepardsized}
%\end{center}
%\end{frame}


\begin{frame}{Persistent Unobserved Heterogeneity}
Suppose we think about a model with a friction such as a switching cost.
\begin{itemize}
\item If $y_{it} \neq y_{i,t-1}$ you pay a switching cost $F_i$.
\item How do we use data to tell apart large switching costs $F_i  \gg0$ from persistent tastes $Cov(\epsilon_{i,t},\epsilon_{i,t-1}) > 0$ ?
\item The \alert{conditional independence assumption} tells us it has to be the switching cost not the autocorrelated unobservables.
\item This is probably why people don't like this assumption.
\end{itemize}
\end{frame}


\begin{frame}{Discrete Unobserved Types}
\begin{itemize}
\item Up until now we consider models satisfying Rust's \alert{conditional independence} assumption on the $\varepsilon$'s. This rules out persistence in unobservables which are economically meaningful.
\item Suppose there are two types of buses good $(s_i=g)$ and bad $(s_i=b)$.
\item Assume that this is known to HZ but not the econometrician.
\item Single period utility now depends on $\alert{s_i}$ so $u(x_{it},s_i,d_{it}; \theta)$ \alert{unobserved state variable}.
\item In case of the nested fixed point algorithm, this unobserved persistent heterogeneity is not a big problem as we can solve for the value function (and expected policy functions) given the state variables and \alert{integrate it out} in the likelihood
\end{itemize}
\end{frame}



\begin{frame}{Unobserved State Variables: What happened?}
\begin{eqnarray*}
\Pr(d_{i1},\ldots,d_{iT} | x_{i1},\ldots,x_{iT} ) &=& \sum_{s}  \prod_{t=1}^T \Pr(d_{it} | x_{it} )  p(s_i) \\
\end{eqnarray*}
\begin{itemize}
\item \alert{Conditional on $s_i$ replacement decisions are independent across $t$ given $x_{it}$}.
\item The resulting likelihood is just a \alert{finite mixture model}.
\item These can be hard to solve when both $s_i$ and its distribution $p(s_i)$ are unknown.
\item Arcidiacono and Miller (2011) provide theoretical results for these types of problems.
\end{itemize}
\end{frame}


\begin{frame}{A much earlier application}
\textbf{Pakes (1986): Patents as Options}\\
 How much are patents worth? Valuable for optimal patent length and design? Sufficient incentive for innovation?
\begin{itemize}
\item $Q_A$: value of patent at age $A$
\item Goal of paper is to estimate $Q_A$ using data on their renewal. $Q_A$ is  inferred from patent renewal process via \alert{revealed preference} for patent renewal behavior.
\item Treat renewal systems as exogenous (in Europe)
\end{itemize}
Timing
\begin{itemize}
\item For $a=1,\ldots,L$ a patent can be renewed by paying the fee $c_a$.
\item At age $a=1$ patent holder gets $r_1$ from patent
\item Decide whether or not to renew (pay $c_1$ and go to $a_2$).
\item At age $a=2$ get $r_2$ from patent
\item and so on...
\end{itemize}

\end{frame}


\begin{frame}{Pakes (1986)}
Gives us the value function
\begin{eqnarray*}
V&\equiv& \max_{t \in [a,L] }\sum_{a'=1}^{L-a} \beta^{a'} R(a+ a') \\
R(a) &=& \begin{cases} r_a -c_a, & \mbox{if } t \geq a \mbox{ when you hold patent} \\0 & \mbox{if } t < a \mbox{ after patent expires} \end{cases}
\end{eqnarray*}
\begin{itemize}
\item $t$ above denotes the age which allows the patent to expire and is the choice variable.  Another \alert{optimal stopping} problem.\\
\item $R(a)$ are the profits from year $a$.  This is a \alert{controlled stochastic process}. It is random but affected by the actions of the agent.
\end{itemize}
\end{frame}


\begin{frame}{Pakes (1986)}
\begin{itemize}
\item The maximum age $L$ is finite so it is finite-horizon DP.
\item The single period revenue $r_a$ is the state variable.
\item We can solve the problem with \textit{backward recursion}.
\end{itemize}
\begin{eqnarray*}
V_a(r_a) = \max  \left\{0, Q_a \equiv r_a + \beta E[V_{a+1} (r_{a+1} ) | \Omega_a] - c_a  \right\}
\end{eqnarray*}
\vspace{-0.75cm}
\begin{itemize}
\item Renew iff $Q_a - c_a > 0$.
\item $\Omega_a$: history up to age $a = \{r_1,r_2,\ldots,r_a\}$.
\item Expectation is over $r_{a+1} | \Omega_{a}$. The sequence of conditional distributions $G_a \equiv F(r_{a+1} | \Omega_a)$, $a=1,2,\ldots$ is an important component of model specification.
\begin{eqnarray*}
r_{a+1} = \begin{cases} 0  & \mbox{w. prob } \exp(-\theta r_a) \\
\max(\delta r_a,z) & \mbox{w. prob } 1-\exp(-\theta r_a) \end{cases}
\end{eqnarray*}
\end{itemize}
\end{frame}


\begin{frame}{Pakes (1986)}
Model has the following parameters
\begin{itemize}
\item density of $z$ $q_a = \frac{1}{\sigma_a} \exp[-(\gamma+z)/\sigma_a]$ and $\sigma_a = \phi^{a-1} \sigma$, for $a=1,\ldots,L-1$. 
\item $(\delta, \theta, \gamma, \phi, \sigma)$ are the structural parameters of the model
\item Break down the model period by period and decide whether or not to renew if $Q_a = r_a + $ ``option value''.
\item Option value is about keeping the patent alive in case it pays off in the future.
\end{itemize}
Implications
\begin{itemize}
\item Drop out at age $a$ if $c_a > Q_a$
\item Optimal decision is characterized by cutoff points $Q_a > c_a \Leftrightarrow r_a > \overline{r}_a$  (Key assumptions is $Q_a$ increasing /single crossing )
\item Cutoff points are increasing sequence $\overline{r}_{a} < \overline{r}_{a+1} < \ldots < \overline{r}_{L-1}$.
\end{itemize}
\end{frame}


\begin{frame}{Estimation}
Instead of using Pakes' notation $r_t$ for the patent revenue. We will use the generic Rust notation of $\epsilon_t$ the unobserved state variable, and $i_t$ to denote the choice (renewal).
\begin{itemize}
\item For a single patent $\tilde{T}$ denotes the age at which it is allowed to expire. Let $T = \min(L-1,\tilde{T})$ denote the period sins which the agent makes a renewal decision where we model the agent's choice.
\item $\epsilon$ follows a first-order Markov process $F(\epsilon' | \epsilon)$
\item Age-specific policy function by $i^*_t(\epsilon)$.
\end{itemize}
Likelihood function is 
\begin{eqnarray*}
\ell(i_1,\ldots,i_T | \epsilon_0,i_0,\theta) = \prod_{t=1}^T \Pr(i_t | i_0, \ldots,  x_{t-1}, i_{t-1} ; \epsilon_0, \theta)
\end{eqnarray*}
Serial correlation in $\epsilon$ means there is dependence among $i_{t}, i_{t-2}$ even after conditioning on $x_{t-1},i_{t-1}$. 
\end{frame}

\begin{frame}{Simulation}
\begin{itemize}
\item It might seem like we were stuck since it no longer has a closed form.  However, we can simulate the ``outer loop'' of the nested fixed point routine given a guess  of $i^*_t(\epsilon,\theta)$.
\item Because $\epsilon$ is serially correlated we need to start with an initial $\epsilon_0$ (or distribution) and assume that it is known.  This is the \alert{initial conditions problem} of finite MDPs.
\item Note that simulation is part of the ``outer loop'' of nested fixed point estimation routine. So at the point when we simulate, we already know the policy functions $i_t^{*}(\epsilon,\theta)$ (How would you compute this?)
\end{itemize}
\end{frame}

\begin{frame}{Naive Frequency Simulator (Don't do this...)}
Go back to the full likelihood function (condition on initial $\epsilon_0$ for serial correlation):
\begin{align*}
\ell(i_1,\ldots,i_T | i_0, \epsilon_0, \theta) = \Pr(i_t^{*} (\epsilon_t,\theta) = i_t, \quad \forall t =1,\ldots,T)
\end{align*}
Need to take probability over distribution of $(\epsilon_1,\ldots,\epsilon_T | \epsilon_0)$.  Let $F(\epsilon_{t+1} | \epsilon_t,\theta)$ then the above probability can be expressed as the integral:
\begin{align*}
\int \cdots \int \prod_t \symbf{1}(i_t^{*}(\epsilon_t,\theta) = i_t) \prod_t d F(\epsilon_t | \epsilon_{t-1}; \theta)
\end{align*}
Simulate by drawing sequences of $(\epsilon_t)$.
\end{frame}


\begin{frame}{Naive Frequency Simulator (Don't do this...)}
Simulate by drawing sequences of $(\epsilon_t)$ and for each draw $s=1,\ldots,S$ we take as initial values $(x_0,i_0,\epsilon_0)$ then
\begin{itemize}
\item Generate $(\epsilon_1^s, i_1^s)$
	\begin{enumerate}
	\item Generate $\epsilon_1^s \sim F(\epsilon_1 | \epsilon_0)$ 
	\item Compute $i_1^s = i_1^{*}(\epsilon_1^s; \theta)$
	\end{enumerate}
\item Generate $(\epsilon_2^s, i_2^s)$
	\begin{enumerate}
	\item Generate $\epsilon_2^s \sim F(\epsilon_2 | \epsilon_1^s)$ 
	\item Subsequently compute $i_2^s = i_2^{*}(\epsilon_2^s; \theta)$
	\end{enumerate}
\item And so on, up to $(\epsilon_T^s,i_T^s)$.
\end{itemize}
\end{frame}


\begin{frame}{Naive Frequency Simulator (Don't do this...)}
And for the case where $(i,x)$ are both discrete (Rust) we can approximate:
\vspace{-0.5cm}
\begin{eqnarray*}
\ell(i_t,\ldots,i_T | \epsilon_0,i_0;\theta) \approx \frac{1}{S} \sum_s \prod_{t=1}^T \symbf{1}(i_t^s = i_t)
\end{eqnarray*}
Frequency of simulated sequences which match observed sequence.  $T$ long or $S$ small you're in trouble (non-smooth).
\end{frame}



\begin{frame}{Importance Sampling: Particle Filtering}
\begin{itemize}
\item We can use importance sampling to simulate the likelihood function. 
\item This is not straightforward given time dependence in $(i_t,\epsilon_t)$
\item Consider particle filtering approach from Fernandez-Villaverde and Rubio-Ramirez (2007) or Flury and Shehard (2008) (non-Gaussian Kalman filtering).
\item A more up to date take: Blevins (2016) : Sequential {Monte Carlo} Methods for Estimating Dynamic Microeconomic Models
\end{itemize}
\end{frame}

\begin{frame}{Importance Sampling: Particle Filtering}
\begin{itemize}
\item Evolution of utility shocks $\epsilon_t | \epsilon_{t-1} \sim f(\epsilon' | \epsilon)$. Ignore dependence of distribution of $\epsilon$ on age $t$ for convenience.
\item As before, the policy function is $i_t = i^{*}(\epsilon_t)$
\item Let $\epsilon^t \equiv \{\epsilon_1,\ldots,\epsilon_t\}$.
\item The initial values of $y_0$ and $\epsilon_0$ are known
\end{itemize}
Go back to the factorized likelihood
\begin{eqnarray*}
\ell(y^T | y_0,\epsilon_0 ) &=& \prod_{t=1}^T \ell (y_t | y^{t-1},y_0,\epsilon_0) = \prod_{t=1} \int \ell (y_t |\epsilon^t, y^{t-1}) p(\epsilon^t | y^{t-1} )d \epsilon^t  \\
&\approx& \frac{1}{S} \sum_s \ell(y_t | \epsilon^{t | t-1,s},y^{t-1})
\end{eqnarray*}
We omit conditioning on $(\epsilon_0, y_0)$ for convenience, and $\epsilon^{t | t-1,s}$ is a simulated draw of $\epsilon^t \sim p(\epsilon^t | y^{t-1})$.
\end{frame}



\begin{frame}{Importance Sampling: Particle Filtering}
Let's look more closely at the last line:
\begin{itemize}
\item first term: $\ell(y_t, | \epsilon^t,y^{t-1})$ we can calculate for a value of $\epsilon_t$
\begin{eqnarray*}
\ell(y_t | \epsilon^t , y^{t-1}) = p(i_t | \epsilon^t, y^{t-1}) = p(i_t | \epsilon_t) = \symbf{1}(i(\epsilon_t) = i_t)
\end{eqnarray*}
\item the second term $p(\epsilon^t | y^{t-1})$ is generally not obtainable in closed form. So numerical integration is not feasible. Particle filtering let's us draw $\epsilon^t$ from this distribution for every period $t$.
\end{itemize}
Particle filtering proposes a recursive approach to draw sequences $p(\epsilon^t | y^{t-1})$ for every $t$
\end{frame}


\begin{frame}{Particle Filtering Algorithm}
\textbf{First period:} $t=1$ In order to simulate the integral corresponding to the first period we need to draw from $p(\epsilon^1 | y^0,\epsilon_0)$ (easy).  
\begin{itemize}
\item We draw $\{\epsilon^{1|0,s }\}_{s=1}^S$ according to $f(\epsilon' | \epsilon_0)$.
\item The notation $\epsilon^{1|0,s}$ makes it explicit that the $\epsilon$ is a draw from $p(\epsilon^1 | y^0,\epsilon_0)$
\item Use the $S$ draws we can evaluate the period $t=1$ likelihood.
\end{itemize}
\textbf{Second period:} $t=2$. We need to draw from $p(\epsilon^2 | y^1)$ factorize as:
\begin{eqnarray*}
p(\epsilon^2 | y^1) = p(\epsilon^1 | y^1) \cdot p(\epsilon_2 | \epsilon^1) \mbox{ recall } \epsilon^2 \equiv \{\epsilon_1,\epsilon_2\}
\end{eqnarray*}
\end{frame}

\begin{frame}{Filtering Step}
Getting a draw from $p(\epsilon^1| y^1)$, given that we already have draws $\{\epsilon^{1|0,s} \}$ from $p(\epsilon^1 | y_0)$, from the previous period $t = 1$, is the heart of particle filtering. We use the principle of importance sampling: by Bayes' Rule
\begin{eqnarray*}
p(\epsilon^1 | y^1) \propto p(y_1 | \epsilon^1, y^0) \cdot p (\epsilon^1 | y^0)
\end{eqnarray*}
Hence, if our desired sampling density is $p(\epsilon^1 | y^1)$, but we actually have draws $\{ \epsilon^{1|0,s}\}$ from $p(\epsilon^1| y^0)$, then the importance sampling weight for the draw
 $\epsilon^{1|0,s}$ is proportional to
\begin{eqnarray*}
\tau_1^s \equiv p(y_1 | \epsilon^{1|0,s},y^0)
\end{eqnarray*}
Note that this coincides with the likelihood contribution for period 1, evaluated at the shock $\epsilon^{1|0,s}$.The SIR algorithm in Rubin (1988) proposes that making $S$ draws with replacement from samples $\{ \epsilon^{1|0,s}\}_{s=1}^S$, using weights proportional $\tau_1^s$ yields draws from the desired density $p(\epsilon^1 | y^1)$ which we denote $\{ \epsilon^{1|0,s}\}_{s=1}^S$.
\end{frame}

\begin{frame}{Prediction Step}
For the second term in the equation: we simply draw one $\epsilon_2^s$ from $f(\epsilon' | \epsilon^{1,s})$, for each draw $\epsilon^{1,s}$ from the filtering step. This is the \textbf{prediction} step.\\
\vspace{0.25cm}
By combining the draws from these two terms, we have $\{ \epsilon^{2|1,s}\}_{s=1}^S$.which is $S$ drawn sequences from $p(\epsilon^2 | y^1)$. Using these $S$ draws, we can evaluate the simulated likelihood for period 2
\end{frame}

\begin{frame}{Prediction Step (Continued)}
\textbf{Third period, $t = 3$:} start again by factoring
\begin{eqnarray*}
p(\epsilon^3 | y^2) = p(\epsilon^2 | y^2) \cdot p(\epsilon^3 | \epsilon^2)
\end{eqnarray*}
As above, drawing from  requires filtering the draws $\{ \epsilon^{2|1,s}\}_{s=1}^S$, from the previous period $t = 2$, to obtain draws $\{ \epsilon^{2,s}\}_{s=1}^S$. Given these draws, draw $\epsilon_3^s \sim f(\epsilon' | \epsilon^{2,s})$ for each $s$.

And so on. By the last period $t = T$, you have
\begin{eqnarray*}
\left\{ \{ \epsilon^{t|t-1,s}\}_{s=1}^S \right\}_{t=1}^T
\end{eqnarray*}
\end{frame}

\begin{frame}{Prediction Step (continued)}
Hence the factorized likelihood can be approximated by simulation as:
\begin{eqnarray*}
\prod_t \frac{1}{S} \sum_s \ell (y_t | \epsilon^{t|t-1,s},y^{t-1})
\end{eqnarray*}
As noted above, the likelihood term $\ell(y_t | \epsilon^{t | t-1,s},y^{t-1})$ coincides with the simulation weight $\tau_t^s$. Hence the simulated likelihood can also be constructed as:
\begin{eqnarray*}
\log \ell (y^T | y_0, \epsilon_0) = \sum_t \log \left\{ \frac{1}{S} \sum_s \tau_t^s \right\}
\end{eqnarray*}
\end{frame}

\begin{frame}{Particle Filtering (Summary)}
\begin{itemize}
\item Start by drawing $\{ \epsilon^{1|0,s}\}_{s=1}^S$ from $p(\epsilon^1 | y^0, \epsilon_0)$.
\item In period $t$, we start with $\{ \epsilon^{t-1|t-2,s}\}_{s=1}^S$ draws from $p(\epsilon^{t-1} | y^{t-2},\epsilon_0)$.
\begin{enumerate}
\item \textbf{Filter step:} Calculate proportion weights $\tau_{t-1}^s \equiv p(y_{t-1} | \epsilon^{t-1 | t-2,s}, y^{t-2}) $ using $p(i_t | \epsilon_t)$. Draw$\{\epsilon^{t-1 | t-1,s}\}_{s=1}^S$ by resampling from $\{ \epsilon^{t-1|t-2,s}\}_{s=1}^S$ with weights $\tau_{t-1}^s$.
\item \textbf{Prediction step:} Draw $\epsilon_t^s$ from $p(\epsilon_t | \epsilon^{t-1 | t-1,s})$ , for $s=1,\ldots,S$. Combine to get 
$\{ \epsilon^{t | t-1,s} \}_{s=1}^S$.
\end{enumerate}
\item Set $t=t+1$ and go back to step 2.  Stop when $t=T+1$.
\end{itemize}
The difference is that the crude simulator draws $S$ sequences and puts zero weight on those which don't match the observed sequence. In each period $t$ we just keep sequences where predicted choices match observed choice of \textit{that period}. This is more accurate likelihood as long as $S$ is large enough that we don't have all the weight on a single sequence in period $t$.
\end{frame}


\begin{frame}{References}
\begin{itemize}
\item Fernandez-Villaverde, J., and J. Rubio-Ramirez (2007): ``Estimating Macroeco- nomic Models: A Likelihood Approach,'' Review of Economic Studies, 74, 1059-1087.\\
\item Flury, T., and N. Shephard (2008): ``Bayesian inference based only on simulated likelihood: particle filter analysis of dynamic economic models,'' manuscript, Oxford University\\
\item Pakes, A. (1986): ``Patents as Options: Some Estimates of the Value of Holding European Patent Stocks,'' Econometrica, 54(4), 755-84.\\
\item Rubin, D. (1988): ``Using the SIR Algorithm to Simulate Posterior Distributions,'' in Bayesian Statistics 3, ed. by J. Bernardo, M. DeGroot, D. Lindley, and A. Smith. Oxford University Press.\\
\end{itemize}
\end{frame}


\end{document}














































